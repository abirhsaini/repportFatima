%%%%%%%%%%%%%%%%%%%%%%%%%%%%%%%%%%%%%%%%%%%%%%%%%%%%%%%%%%%%%%%
%
% Welcome to Overleaf --- just edit your LaTeX on the left,
% and we'll compile it for you on the right. If you open the
% 'Share' menu, you can invite other users to edit at the same
% time. See www.overleaf.com/learn for more info. Enjoy!
%
%%%%%%%%%%%%%%%%%%%%%%%%%%%%%%%%%%%%%%%%%%%%%%%%%%%%%%%%%%%%%%%
\documentclass[12pt,a4paper,oneside]{book}


\input{Packages}


%\makenoidxglossaries 

%\newglossaryentry{latex}
%{
       % name=latex,
        %description={Is a mark up language specially suited for scientific documents}
%}



%\selectlanguage{English}


\hypersetup{
    colorlinks=true,
    linkcolor=black,
    filecolor=magenta,      
    urlcolor=cyan,
    pdfauthor={FatimaTaoufiq},  %put your name here
    pdftitle={Rapport PFE Fatima Taoufiq},  %PDF title
    pdfpagemode=FullScreen,
    }



\begin{document}



\definecolor{codegreen}{rgb}{0,0.6,0}
    \definecolor{codegray}{rgb}{0.5,0.5,0.5}
    \definecolor{codepurple}{rgb}{0.58,0,0.82}
    \definecolor{backcolour}{rgb}{0.95,0.95,0.92}
    
    \lstdefinestyle{mystyle}{
        backgroundcolor=\color{backcolour},   
        keywordstyle=\color{magenta},
        numberstyle=\tiny\color{codegreen},
        stringstyle=\color{codepurple},
        basicstyle=\ttfamily\footnotesize,
        breakatwhitespace=false,         
        breaklines=true,                 
        captionpos=b,                    
        keepspaces=true,                 
        numbers=left,                    
        numbersep=5pt,                  
        showspaces=false,                
        showstringspaces=false,
        showtabs=false,                  
        tabsize=2
    }




%\include{Frontmatter/Page_de_garde} %FRENCH ONE


\thispagestyle{empty}
\includegraphics[scale=0.08]{Logos/Logo_INPT.png} 
         \hspace{11cm}  
         \includegraphics[scale=0.6]{Logos/SQLI_LOGO.png}
        
\vspace{0.4cm}
\begin{center}
{\large \textsc{\textbf{Mémoire du projet de fin d'études}}}\\[0.1cm]
{\large {Pour l’obtention du Diplôme d’Ingénieur d’État en Télécommunications 
et Technologies de l’Information}}\\[0.1cm]
{\large Filière:\textsc{\textit{\textbf{ Advanced Software Engineering for Digital Services (A.S.E.D.S)}}}} \\[0.05cm] 
\vspace{0.4cm}
\vspace{-0.04cm}
% Title
\rule{\linewidth}{0.3mm} \\[0.3cm]   % à ajuster l'éspace en cas de besoin: [1cm]
 { \huge \textbf{Amélioration d'une Plateforme E-commerce : Intégration de Payconiq}} \\[0.15cm] 
\rule{\linewidth}{0.3mm} \\[0.3cm]


  %change the scale to suit your logo

\vspace{0.8cm}

% Author and supervisor
\noindent
\begin{minipage}{0.9\textwidth}
    \vspace{-7mm}
  \begin{flushleft} \large
    \emph{Réalisé par :}\\
    Mme. TAOUFIQ Fatima %\& Mr/Mrs/Ms. Xxxx \textsc{XXXX}  %Au cas de binôme, remove the % \\
  \end{flushleft}
\end{minipage}
\begin{minipage}{0.4\textwidth}

\end{minipage}\\[0.3cm]

{\large \textit{Soutenu le X Septembre 2024, devant les membres de jury : }}\\[0.3cm]


\begin{tabular}{p{1cm}lll}
  & \large Pr. MARGHOUBI Rabia & \large INPT & \large - Encadrante interne  \\[0.1cm]
  & \large Pr. HAFIDDI Hatim & \large INPT & \large - Examinateur \\[0.1cm]
  & \large Pr. RADGUI Amina & \large INPT & \large - Examinatrice  \\[0.1cm]
  & \large Mme. ELJABARI Sara & \large SQLI & \large - Encadrante externe  \\[0.1cm]
  & \large M. JIRARI Adil & \large SQLI & \large - Encadrant externe  \\[0.1cm]


\end{tabular}

\includegraphics[scale=0.65]{Logos/ZLAFA.png}


\textsc{Agence Nationale de Réglementation des Télécommunications}\\
\textsc{Institut National des Postes et Télécommunications}
% Bottom of the page

%\vspace{0.3cm}
{\large Année universitaire : 2023/2024}
   
\end{center}


 % ENGLISH

\afterpage{\blankpage}  %page vide obligatoire

\frontmatter


\chapter*{Dédicace}

\addcontentsline{toc}{chapter}{Dédicaces}


\begin{figure}[h]
    \centering
    \includegraphics[width=19cm]{Figures/dedicace.png} 
\end{figure}





\chapter*{Remerciements}
\addcontentsline{toc}{chapter}{Remerciements}



Avant tout, je remercie Allah, le Tout-Puissant, de m'avoir accordé le courage et la patience nécessaires pour mener ce travail à terme dans des trés bonnes conditions.

\vspace{10pt}
Je souhaite exprimer ma gratitude à toutes les personnes qui, par leur soutien ou par leur simple présence, ont contribué à rendre mon travail à la fois instructif, bénéfique et agréable.

\vspace{10pt}
Je souhaite exprimer ma profonde gratitude à mon encadrante, \textbf{Pr. Rabiaa MARGHOUBI}, dont les connaissances, le savoir-faire et les précieuses orientations ont grandement facilité mon travail. Je lui suis reconnaissante pour ses conseils avisés, son suivi attentif, ainsi que pour la fierté et l'ambition que j'ai pu développer grâce à son soutien intensif et son aide précieuse.

\vspace{10pt}
Je souhaite également remercier chaleureusement mes encadrants externes, \textbf{Mme ELJABARI Sara} et \textbf{M. JIRARI Adil}, pour leur confiance, leur collaboration et leur soutien tout au long de mon stage. Leur supervision, leurs recommandations éclairées, leurs orientations précieuses et leur rigueur m'ont été d'une aide précieuse et ont facilité mon intégration. Je tiens aussi à adresser mes sincères remerciements aux membres de l'entreprise SQLI Maroc pour l'expérience enrichissante et captivante qu'ils m'ont offerte pendant ces mois de stage parmi eux.

\vspace{10pt}
Je tiens à exprimer ma sincère gratitude à mes examinateurs, \textbf{Pr. HAFIDDI Hatim} et \textbf{Pr. RADGUI Amina}, pour leur évaluation et leurs précieux retours. Je remercie également l'ensemble du corps enseignant de l’Institut National des Postes et Télécommunications pour leur soutien, leur expertise et leur engagement, qui ont grandement contribué à l’enrichissement de mon parcours académique.


% 

\chapter*{Résumé}
\addcontentsline{toc}{chapter}{Résumé}

Dans un contexte où le commerce électronique est en constante évolution, les entreprises doivent régulièrement mettre à jour et optimiser leurs plateformes pour rester compétitives. L'accélération des innovations technologiques et les attentes croissantes des consommateurs exigent une adaptation continue pour offrir des expériences utilisateur de haute qualité et répondre aux nouvelles exigences du marché. \\


Ce rapport présente un aperçu concis de mes contributions lors de mon stage de fin d'études effectué au sein de SQLI Maroc, dans le but d'obtenir le titre d'Ingénieur d'État en Télécommunications et Technologies de l'Information, spécialisation en Ingénierie de développent des services numériques à l'Institut National des Postes et Télécommunications. Ce projet avait pour objectif d'améliorer une plateforme e-commerce existante pour un client de luxe, en intégrant la méthode de paiement Payconiq spécifiquement pour le marché belge, tout en corrigeant divers bugs afin d'optimiser la performance du système.\\

En prenant en compte la complexité de SAP Hybris, la réalisation de ce projet a nécessité l'utilisation de technologies telles que JEE et Spring pour le développement, ainsi que JUnit pour le testing. L'intégration de Payconiq a été un défi majeur, nécessitant une attention particulière à la compatibilité et à la performance pour garantir une solution adaptée aux besoins spécifiques du marché belge.

\noindent\rule[2pt]{\textwidth}{0.5pt}

{\textbf{Mots clés :}}
E-commerce, Payconiq, SAP Hybris, JUnit, JEE, Spring, Déboggage
\\
\noindent\rule[2pt]{\textwidth}{0.5pt}

% % \cleardoublepage
% %

% \chapter*{Abstract}
\addcontentsline{toc}{chapter}{Abstract}

In a context where e-commerce is constantly evolving, companies must regularly update and optimize their platforms to remain competitive. The acceleration of technological innovations and the growing expectations of consumers demand continuous adaptation to deliver high-quality user experiences and meet new market requirements.
\vspace{10pt}

This report provides a concise overview of my contributions during my internship at SQLI, aiming to obtain the title of State Engineer in Telecommunications and Information Technologies, with a specialization in Advanced Software Engineering for Digital Services from the National Institute of Posts and Telecommunications. This project aimed to enhance an existing e-commerce platform for a client by integrating the Payconiq payment method specifically for the Belgian market, while also fixing various bugs to optimize system performance.
\vspace{10pt}

Considering the complexity of SAP Hybris, the realization of this project required the use of technologies such as JEE and Spring for development, as well as JUnit for testing. The integration of Payconiq was a major challenge, requiring particular attention to compatibility and performance to ensure a solution tailored to the specific needs of the Belgian market.
\vspace{10pt}

\noindent\rule[2pt]{\textwidth}{0.5pt}

{\textbf{Key Words :}}
E-commerce, Payconiq, SAP Hybris, JUnit, JEE, Spring, Debugging.
\\
\noindent\rule[2pt]{\textwidth}{0.5pt}

% \cleardoublepage
%

% \chapter*{\RL{ملخص}}
\addcontentsline{toc}{chapter}{Résumé arabe}

\begin{RLtext}
 \noindent\hspace{10pt} 
 في ظل التطور المستمر الذي يشهده مجال التجارة الإلكترونية، تجد الشركات نفسها مضطرة لتحديث وتحسين منصاتها بشكل منتظم للحفاظ على قدرتها التنافسية. إذ أن تسارع الابتكارات التكنولوجية وتزايد توقعات المستهلكين يفرضان تكيفًا مستمرًا من أجل تقديم تجارب استخدام عالية الجودة وتلبية المتطلبات الجديدة للسوق.

 \vspace{10pt}

 لذا فهذا التقرير يقدم نظرة موجزة عن مساهماتي التي قدمتها في شركة \LR{SQLI Maroc} خلال فترة التدريب النهائي لنيل شهادة مهندس دولة في الاتصالات وتكنولوجيا المعلومات، بتخصص في هندسة البرمجيات المتقدمة للخدمات الرقمية من المعهد الوطني للبريد والمواصلات. كان الهدف من هذا المشروع تحسين منصة تجارة إلكترونية قائمة لأحد العملاء، من خلال دمج نظام الدفع \LR{Payconiq} خصيصًا للسوق البلجيكي، مع تصحيح مجموعة من الأخطاء البرمجية لتحسين أداء النظام.
 \vspace{10pt}

 ونظراً لتعقيد منصة \LR{SAP Hybris}، تطلب تنفيذ هذا المشروع استخدام تقنيات مثل \LR{JEE} و\LR{Spring} للتطوير، بالإضافة إلى \LR{JUnit} للاختبار. شكلت عملية دمج \LR{Payconiq} تحديًا رئيسيا، حيث استلزمت اهتماما خاصا بالتوافق والأداء لضمان تقديم حل يلبي الاحتياجات المحددة للسوق البلجيكي.

 \vspace{10pt}
\end{RLtext}

\noindent\rule[2pt]{\textwidth}{0.5pt}

\noindent
\begin{RLtext}
    \textbf{الكلمات المفتاحية\LR{:}} التجارة الإلكترونية،\LR{Payconiq}، \LR{SAP Hybris}، \LR{JUnit}، \LR{JEE}، \LR{Spring}، \LR{debugging}.
\end{RLtext}

\noindent\rule[2pt]{\textwidth}{0.5pt}


% \cleardoublepage
%


% \chapter{Liste des sigles et acronymes}

\begin{tabbing}
    \hspace{3cm} \= \hspace{7cm} \= \kill

    \textbf{AMER} \> America \\

    \textbf{API} \> Application Programming Interface \\

    \textbf{APAC} \> Asia-Pacific, Australia, and Canada \\

    \textbf{ANT} \> Another Neat Tool \\

    \textbf{DAO} \> Data Access Object \\

    \textbf{DEV} \> Development \\

    \textbf{EMEA} \> Europe, the Middle East, and Africa \\

    \textbf{HTML} \> Hypertext Markup Language \\

    \textbf{HAC} \> Hybris Administration Console \\

    \textbf{HMC} \> Hybris Management Console \\

    \textbf{HTTP} \> Hypertext Transfer Protocol \\

    \textbf{HTTPS} \> Hypertext Transfer Protocol Secure \\

    \textbf{INT} \> Integration \\

    \textbf{JEE} \> Java Enterprise Edition \\

    \textbf{JSE} \> Java Standard Edition \\

    \textbf{MEP} \> Mise En Production \\
    
    \textbf{OAT} \> Operational Acceptance Testing \\

    \textbf{OMS} \> Order Management System \\

    \textbf{PIM} \> Product Information Management \\

    \textbf{PRD} \> Production \\

    \textbf{PSD2} \> Payment Services Directive 2 \\

    \textbf{QA} \> Quality Assurance \\

    \textbf{SCA} \> Strong Customer Authentication \\

    \textbf{SAP} \> Systems, Applications, and Products \\

    \textbf{SGBD} \> Système de Gestion de Bases de Données \\

    \textbf{TFS} \> Team Foundation Server \\

    \textbf{UAT} \> User Acceptance Testing \\

    \textbf{UML} \> Unified Modeling Language \\


\end{tabbing}




\listoffigures
\addcontentsline{toc}{chapter}{Table des figures} 
%figures are added automatically here

\listoftables
\addcontentsline{toc}{chapter}{Liste des tableaux} 
%tables are added automatically here



\tableofcontents
\addcontentsline{toc}{chapter}{Table des matières}
%contents are added automatically here



\mainmatter

% \chapter*{Introduction générale}
\addcontentsline{toc}{chapter}{Introdcution}


Dans un monde avec une grande évolution, où les technologies changent rapidement, la nécessité d'une formation continue efficace, simple et accessible est devenue primordiale pour rester présent sur le marché du travail. Les entreprises et les individus reconnaissent de plus en plus l’importance d'une formation continue en certains domaines pour améliorer leurs compétences, acquérir de nouvelles connaissances et s'adapter aux changements rapides de l'environnement professionnel.\\

C'est dans ce contexte que s'inscrit notre projet de fin d'études (PFE), qui vise à créer une plateforme de formation continue adaptée aux besoins actuels des employés et des clients de 4D. L’objectif principal de notre projet est de fournir une solution technologique robuste et conviviale qui facilite l'accès aux différentes formations du langage 4D ainsi que les autres langages, tout en offrant des fonctionnalités avancées pour suivre, évaluer et personnaliser le processus d'apprentissage.\\

Ce rapport présente le processus de développement de notre plateforme de formation continue, composé de quatre chapitres. Dans le premier chapitre, nous exposons le contexte général du projet, à savoir, la présentation de l’organisme d’accueil 4D Logiciels, le contexte général et la conduite du projet. Le deuxième chapitre est dédié à l’analyse et à la spécification des besoins pour développer cette plateforme. Le troisième chapitre porte sur la définition des architectures utilisées, ainsi que la modélisation des diagrammes de classes et de séquence. Enfin, dans le quatrième chapitre, nous abordons l’implémentation et la validation de la solution, suivies par une conclusion générale où nous discutons des perspectives d’évolution.\\


%debut of chapters

\chapter{Contexte général du projet}
\label{chap:Contexte général du projet}



Ce chapitre situe mon projet de fin d’études dans son environnement organisationnel et contextuel. Il commence par une présentation de l’organisme d’accueil, SQLI Maroc. Ensuite, il détaille la problématique ayant conduit à la réalisation de ce projet ainsi que les objectifs visés. Enfin, la méthodologie adoptée pour mener à bien le projet est abordée.

\newpage


\section{Présentation de l’entreprise d’accueil SQLI}

Cette section initiale met en lumière le Groupe SQLI en mettant l’accent sur ses activités clés, son
chiffre d’affaires ainsi que ses clients. EEnsuite, l'accent sera mis sur SQLI Maroc, en mettant en avant ses valeurs fondamentales.

\subsection{Groupe SQLI}

\begin{figure}[h]
    \centering
    \includegraphics[scale=0.7]{Logos/SQLI_LOGO.png} % Replace with the actual filename of the IBM logo image
    \caption{Logo de SQLI \cite{SQLI}}
    \label{fig:LogoSQLI}
\end{figure}

SQLI est une entreprise européenne de services numériques fondée en 1990 par Jean Rouveyrol et Alain Lefebvre. Elle se spécialise dans la conception, le développement et le déploiement de solutions digitales
visant à créer des expériences unifiées \cite{SQLI}. Avec un effectif de 2400 collaborateurs répartis dans 13 pays,
SQLI bénéficie d’une présence internationale solide.

Le succès de SQLI Digital Experience repose sur des valeurs fondamentales telles que la créativité, l'engagement et l'audace visionnaire. Ces valeurs imprègnent chaque aspect de l'entreprise, permettant de repousser les frontières de l'innovation et de concevoir des expériences digitales uniques et captivantes. \cite{valeurSQLI}

\subsubsection{Activités du groupe}

Le groupe SQLI propose une gamme étendue de services pour accompagner les entreprises dans leur transformation numérique. Il inclut l'e-commerce, créant et optimisant des plateformes de vente en ligne performantes. Il offre également des plateformes d'expérience, conçues pour offrir des interactions utilisateur exceptionnelles. En matière de technologie et de transformation, il aide les entreprises à moderniser leurs infrastructures et leurs processus. Ses services de data et insights permettent d'exploiter les données de manière stratégique, tandis que son expertise en marketing digital et design améliore la visibilité et l'attrait des marques. Enfin, son conseil digital guide les entreprises dans l'élaboration et la mise en œuvre de leur stratégie numérique globale, assurant ainsi une transformation digitale réussie. \cite{SQLI}

\subsubsection{Chiffres Clés du groupe}

Les chiffres clés suivants présentent la situation actuelle de SQLI :

\begin{itemize}
    \item[$\bullet$] Fort de 33 ans d’expérience et d’innovation, SQLI fonde son développement sur une expertise technologique de pointe et une politique de veille intensive.
    \item[$\bullet$] SQLI emploie plus de 2400 collaborateurs répartis dans 13 pays, notamment la France, l'Angleterre, la Suède, les Pays-Bas, l'Espagne, l'Allemagne, la Belgique, le Luxembourg, la Suisse et le Maroc.
    \item[$\bullet$] En 2022, le groupe SQLI a atteint un chiffre d’affaires de 251,2 millions de dollars. Ce succès est le résultat d'une offre bien alignée sur les attentes du marché et d'une reprise progressive de la demande de services informatiques.
    \vspace{0.5cm}
\end{itemize}

\begin{figure}[H]
    \centering
    \includegraphics[width=10cm]{Figures/cle chiffre.png} % Replace with the actual filename of the IBM logo image
    \caption{Chiffre Clés de SQLI}
\end{figure}

\subsubsection{Clients du groupe}

SQLI collabore avec une vaste gamme de clients provenant de divers secteurs, y compris l'automobile, la distribution, la banque et l'assurance, le luxe et la mode, la santé, l'industrie et l'énergie, ainsi que les télécommunications. Les grandes entreprises internationales et les organisations locales font appel à SQLI pour ses solutions digitales innovantes, allant de l'optimisation des plateformes d'e-commerce à la transformation numérique des services financiers, en passant par la création d'expériences utilisateur uniques pour les marques de luxe et la digitalisation des processus industriels. Grâce à sa capacité à répondre aux besoins spécifiques de chaque secteur, SQLI bâtit des partenariats solides et durables avec ses clients (voir \textit{Figure~\ref{fig:client}})

\begin{figure}[H]
    \centering
    \includegraphics[width=15cm]{Figures/sqli partenaire .png} % Replace with the actual filename of the IBM logo image
    \caption{Clients du SQLI \cite{SQLI}}
    \label{fig:client}
\end{figure}


\subsection{SQLI Maroc}

SQLI Maroc, créée en 2003 à Rabat par Eric Chanal, représente le centre de Delivery et d'Innovation du Groupe SQLI. Bénéficiant d'une solide expertise et d'une grande expérience, l'entreprise est présente sur trois sites stratégiques : Rabat, où j'ai eu l'opportunité d'effectuer notre stage PFE, Oujda et Casablanca. Le tableau suivant représente sa fiche technique:



% l'ajoute de titre de tableau avec space between titre de tab et le tab
% \captionsetup{type=table}

% \vspace{0.3cm}
% tab
\begin{center}
    \captionsetup{type=table}
    \label{tab:fiche}
    \vspace{0.3cm}
    \begin{tabularx}{17cm}{|X|X|}
      \hline
     \textbf{Dénomination sociale}  & \textbf{SQLI Digital Experience} \\
      \hline
     {Année de fondation} & 2003  \\
      \hline
      {Fondateur} & Eric Chanal  \\
      \hline
     {Siège social} & Rabat, Maroc\\
     \hline  
     {Activité} & Conseil en systèmes et logiciels informatiques.\\
      \hline  
     {Effectif des employés} & Plus de 900 collaborateurs.\\
      \hline
     {Sites d'implantation} & Rabat, Oujda et Casablanca.\\
      \hline
    \end{tabularx}
    \captionof{table}{Fiche technique de SQLI Maroc}
    \end{center}
    

  SQLI Maroc comprend principalement deux structures essentielles, présentées dans la \textit{Figure~\ref{fig:departement}}, à savoir :

\begin{enumerate}
    \item \textbf{SQLI WAX INTERRACTIVE} : accompagne les clients dans leur transition vers la digitalisation afin de renforcer leur positionnement sur le marché. Cette entité intervient principalement sur le plan stratégique en collaborant étroitement avec les clients.
     \item \textbf{SQLI ENTREPRISE} : Cette entité est chargée de la mise en œuvre des systèmes d'information pour les clients. Elle se compose de plusieurs Business Units spécialisées dans différents domaines :
\begin{itemize}

    \item \textbf{E-commerce/ JAVA EE} : Se focalise sur la création et la mise en place de sites de e-commerce ainsi que sur le développement d’applications utilisant la technologie Java EE.
    \item \textbf{Mobile/Front} : Se spécialise dans le développement d’applications mobiles et l’interface utilisateur (front-end) pour les clients.
    \item \textbf{Microsoft} : S'occupe de la réalisation d’applications basées sur les technologies Microsoft.
    \item \textbf{Agency} : Joue un rôle transversal en assurant la conception de l’interface utilisateur (front-end) pour toutes les autres Business Units.
    \item \textbf{Delivery} : Se charge de la gestion des livraisons et des recettes auprès des clients.
\end{itemize}
\end{enumerate}
\vspace{0.5cm}
\begin{figure}[H]  
  \centering  
  \includegraphics[width=14cm]{Figures/departement.png}
  \caption{Départements de SQLI}
  \label{fig:departement}
\end{figure}


Mon stage de fin d'études s'est déroulé dans le département Java JEE, qui regroupe plusieurs projets destinés à de grandes entreprises clientes.


\section{Présentation du projet}

\subsection{Cadre du projet et problématique}

Dans le cadre d'un projet e-commerce pour un client, l'objectif est d'améliorer et d'optimiser sa plateforme actuelle. Il est indispensable de mettre à jour régulièrement cette plateforme, qui joue un rôle crucial dans les activités commerciales en ligne du client, afin de maintenir sa compétitivité et de répondre aux exigences du marché.

Pour cette amélioration, le travail inclut la correction de divers bugs qui affectent la performance et la fiabilité du système. La résolution de ces bugs est cruciale pour garantir une expérience utilisateur fluide et sans interruptions.

En parallèle, l'intégration de nouvelles fonctionnalités est nécessaire pour enrichir l'offre de la plateforme. L'un des changements majeurs est l'intégration de la méthode de paiement Payconiq, destinée spécifiquement au marché belge. Différents défis se posent lors de cette intégration, tels que la compatibilité avec l'architecture existante, la gestion des dépendances et l'assurance que cette nouvelle fonctionnalité ne provoque pas de régressions ou de nouveaux bugs.

L'enjeu majeur consiste donc à corriger les bugs existants tout en intégrant Payconiq de manière efficace, en maintenant la stabilité et la performance globale de la plateforme.

 \subsection{Objectifs du projet}

 Dans le cadre de ce projet, je participerai activement aux diverses activités de l'équipe, contribuant à la fois au développement des fonctionnalités demandées par le client et à l'amélioration continue du système. Les objectifs spécifiques de mon intervention sont les suivants :

 \begin{itemize}
    \item[$\bullet$] \textbf{Intégrer un nouveau mode de paiement, Payconiq, pour le marché belge :} 
    
    Analyser et comprendre l'architecture existante pour intégrer Payconiq, tout en gérant les dépendances et en garantissant la compatibilité avec les autres modules de la plateforme, conformément aux spécificités techniques du marché belge. Cela inclut la configuration, le développement, et des tests rigoureux pour garantir une intégration fluide dans les différents flux de paiement existants.

    \item[$\bullet$] \textbf{Assurer les livraisons dans différents environnements (DEV, INTx, UAT, PRD) :} 
    
    Garantir le bon fonctionnement du code dans chacun de ces environnements, conformément aux exigences de stabilité et de performance. Cela comprend des tests approfondis pour s'assurer que les nouvelles fonctionnalités et intégrations sont stables et opérationnelles avant la mise en production.

    \item[$\bullet$] \textbf{Respecter les meilleures pratiques, les normes, et l'architecture du projet :} 
    
    Assurer la cohérence du code et la stabilité du projet en adhérant aux normes et pratiques de développement établies. Cela inclut le respect des principes d'architecture définis, l'application des bonnes pratiques de codage, et la proposition de solutions conformes aux standards en vigueur au sein de l'équipe.

    \item[$\bullet$] \textbf{Analyser et corriger les bugs détectés :}
    
    Assurer la maintenance corrective du système en identifiant, analysant, et résolvant les bugs remontés par les équipes ou découverts lors des tests, afin de minimiser leur impact sur le fonctionnement global de la plateforme.

\end{itemize}

 %%%%%%%%%%%%%%%%%%%% SECTION 4 %%%%%%%%%%%%%%%%%%%%%%%
\section{Conduite de projet}

\subsection{Présentation des Équipes du Projet}

Après ma période de formation, j’ai intégré une première équipe en tant que stagiaire backend. Cette équipe était responsable des aspects liés à la recherche, aux lunettes et à la mode pour le projet Chanel. À mon arrivée, l'équipe se concentrait sur le Plan 99.5, visant à analyser les bugs, nettoyer les logs, corriger les erreurs et refactoriser le code. La structure de cette équipe est illustrée dans la \textit{Figure~\ref{fig:structure_seasonal_event}}.

Souhaitant approfondir mes compétences dans des domaines complémentaires, j'ai ensuite rejoint une autre équipe, chargée de l’intégration des nouvelles méthodes de paiement pour les différents marchés du client. La structure de cette équipe est illustrée dans la \textit{Figure~\ref{fig:structure_payment}}.

Ces deux équipes font partie d'un projet plus vaste comprenant 15 équipes fonctionnelles différentes. Chaque équipe est composée d'un Scrum Master, d'un expert technique, d'un Product Owner, d'un développeur frontend et de deux responsables qualité, favorisant ainsi le développement d'une solution robuste et performante.

\begin{center}
    \centering
    \includegraphics[width=19cm]{Figures/Seasonal Event Team.png}
    \captionof{figure}{Structure de l'équipe Seasonal Event}
    \label{fig:structure_seasonal_event}
\end{center}

\begin{center}
    \centering
    \includegraphics[width=19cm]{Figures/Cart Checkout & Payment.png}
    \captionof{figure}{Structure de l'équipe Cart Checkout \& Payment}
    \label{fig:structure_payment}
\end{center}


\subsection{Méthodologie de travail : Scrum}

Pour assurer une collaboration efficace au sein de l'équipe, nous avons opté pour la méthodologie Scrum, qui se caractérise par une approche itérative et incrémentale. Scrum nous permet de diviser le travail en sprints, des cycles de développement courts et cadencés, généralement de deux à quatre semaines. À la fin de chaque sprint, une version potentiellement livrable du produit est présentée, ce qui favorise la flexibilité et l'adaptation aux changements. Grâce à cette méthode, nous pouvons rester réactifs et ajuster rapidement notre travail en fonction des évolutions des besoins métiers. En intégrant les retours d'expérience du client à chaque itération, nous assurons une satisfaction optimale de ses attentes. Les rôles bien définis, tels que le Scrum Master, le Product Owner, et l'équipe de développement, garantissent une communication claire et une responsabilité partagée. Cette approche nous permet d'être efficaces tout en maintenant un rythme de travail soutenu et structuré.

\subsection{Outils de collaboration}

\subsubsection{Skype}

\begin{figure}[h]
    \centering
    \includegraphics[scale=0.02]{Logos/Skype-Logo.png} % Replace with the actual filename of the IBM logo image
    \caption{Skype Logo}
\end{figure}

Skype est un outil de communication essentiel qui a facilité le déroulement de mon projet. Il permet d'effectuer des appels téléphoniques et vidéo via Internet, ainsi que le partage d'écran. Pendant toute la durée de mon projet de fin d'études, Skype a joué un rôle crucial en me permettant de maintenir un contact régulier avec mon encadrant, mes collègues et les membres de l'entreprise.

Les appels gratuits entre utilisateurs ont facilité les échanges quotidiens, tandis que la fonction de partage d'écran a été particulièrement bénéfique pour réaliser des présentations et des démonstrations de mon travail en temps réel. Cette fonctionnalité a grandement contribué à la fluidité de nos échanges et à l'efficacité de la communication, éléments essentiels pour la réussite du projet.

En somme, Skype a permis une collaboration efficace et une gestion optimisée des tâches, soutenant ainsi l'avancement et la qualité du projet de fin d’études.

\subsubsection{Git}

\begin{figure}[H]
    \centering
    \includegraphics[scale=0.5]{Logos/git.png}
    \caption{Logo Git}
\end{figure}

Pour la gestion des versions du code source, nous avons opté pour l'utilisation de Git. Git est un système de contrôle de version décentralisé, largement adopté dans le domaine du développement logiciel.

Ce logiciel libre, sous licence GNU, permet de suivre les modifications apportées au code source tout au long du projet. Grâce à Git, nous avons pu gérer efficacement les différentes versions du code, collaborer en simultané avec plusieurs membres de l'équipe, et assurer une traçabilité précise des changements effectués.

\subsubsection{GitLab}


\begin{figure}[H]
    \centering
    \includegraphics[scale=0.1]{Logos/gitlab.jpg}
    \caption{Logo GitLab}
\end{figure}

Pour compléter l'utilisation de Git, nous avons également choisi GitLab comme plateforme d'hébergement de code. GitLab offre des outils puissants pour le contrôle de version et la collaboration en équipe. Il permet à tous les membres de l'équipe de travailler ensemble de manière fluide, quel que soit leur emplacement. En centralisant les dépôts de code, GitLab facilite la gestion des contributions, le suivi des problèmes, et les demandes de fusion (merge requests), renforçant ainsi la coordination et l'efficacité du travail collectif.

La figure suivante (\textit{Figure~\ref{fig:client}}) représente une aperçu globale sur l'espace Gitlab 

\section*{Conclusion}
Au cours de ce chapitre, nous avons mis l’accent sur le périmètre de notre projet. Nous avons éclairé
la méthodologie et le planning suivis pour mener ce projet. Nous entamerons dans le chapitre suivant
la phase d’analyse et spécification du système à développer au cours de laquelle nous comprenons en
profondeurs les besoins utilisateurs et construisons ainsi un système qui y répond.




\chapter{Analyse et spécification des besoins}
\label{chap:Analyse et spécification des besoins}

Ce chapitre présente l'analyse de l'existant et la spécification des besoins pour l'intégration de Bancontact by Payconiq. Nous examinerons les méthodes de paiement actuelles, les spécificités du marché belge, l'architecture existante, ainsi que les besoins fonctionnels et non fonctionnels.
\pagebreak

\section{Étude de l’existant}
\subsection{Méthodes de paiement actuelles}
Le site e-commerce de la marque propose actuellement une gamme diversifiée de méthodes de paiement reconnues mondialement, comprenant :
\begin{itemize}
    \item Visa
    \item MasterCard
    \item PayPal
    \item Klarna Pay Now
    \item Klarna Pay Later
    \item Chanel Gift Card
\end{itemize}
Bien que ces options répondent efficacement aux besoins d'une clientèle internationale, elles ne tiennent pas compte des spécificités locales de certains marchés clés, en particulier celui de la Belgique.
\subsection{Particularités du marché belge}
En Belgique, Bancontact s'est imposé comme l'une des méthodes de paiement privilégiées. Initialement conçu comme un système de paiement par carte de débit national, Bancontact est devenu un élément incontournable du paysage financier belge. Cette solution offre aux consommateurs belges la possibilité d'effectuer des paiements directs depuis leur compte bancaire, que ce soit en magasin, en ligne ou via une application mobile.
Face à l'évolution rapide des technologies et des attentes des consommateurs, Bancontact a réalisé une fusion stratégique avec Payconiq, une solution de paiement mobile innovante. Cette alliance a donné naissance à Bancontact by Payconiq, offrant aux utilisateurs belges une solution de paiement intégrée couvrant à la fois les transactions par carte et les paiements mobiles via une application dédiée.
L'adoption massive de cette solution en Belgique en fait un élément incontournable pour tout e-commerce aspirant à s'implanter solidement sur ce marché. Les chiffres parlent d'eux-mêmes : en 2023, près de 2 millions de Belges ont utilisé Payconiq pour leurs paiements mobiles, soulignant l'importance cruciale de cette méthode dans l'écosystème des paiements locaux.
\subsection{Justification de l'intégration de Bancontact by Payconiq}
L'intégration de Bancontact by Payconiq sur notre plateforme e-commerce présente plusieurs avantages stratégiques majeurs, particulièrement pour conquérir et fidéliser la clientèle belge :
\begin{itemize}
    \item [$\bullet$]\textbf{Adoption généralisée :} Avec une base d'environ 2 millions d'utilisateurs Payconiq et une forte pénétration de Bancontact dans les habitudes de paiement quotidiennes, cette solution est profondément ancrée dans le comportement des consommateurs belges.
    \item [$\bullet$]\textbf{Simplicité et ergonomie :} L'application "Payconiq by Bancontact" offre une expérience de paiement fluide et intuitive, reposant sur un simple scan de QR code, ce qui optimise considérablement le parcours client.
    \item [$\bullet$]\textbf{Sécurité renforcée :} La synergie entre les systèmes Bancontact et Payconiq garantit un niveau de sécurité optimal pour les transactions, s'appuyant sur des protocoles de sécurité robustes et éprouvés.
    \item [$\bullet$]\textbf{Interopérabilité bancaire :} Le support étendu de Payconiq par les principales institutions bancaires belges facilite son adoption et renforce la commodité pour les clients, en centralisant leurs opérations financières.
    \item [$\bullet$]\textbf{Essor des paiements mobiles :} Face à la croissance exponentielle des paiements mobiles en Belgique, l'intégration de solutions comme Payconiq s'avère cruciale pour capter et fidéliser une clientèle, particulièrement auprès des jeunes générations habituées aux transactions via smartphone.
\end{itemize}
\subsection{Architecture du processus de paiement actuel}
Le processus de paiement en place, présenté dans la figure \ref{fig:processus}, repose sur l'interaction harmonieuse de plusieurs systèmes interdépendants, chacun jouant un rôle déterminant dans la sécurisation des transactions et l'optimisation du traitement des commandes. 
\begin{center}
    \centering
    \includegraphics[width=19cm]{Figures/processus.jpeg}
    \captionof{figure}{Processus du paiement}
    \label{fig:processus}
\end{center}
Les composants clés du système sont les suivants :
\begin{itemize}
    \item [$\bullet$]\textbf{Hybris :} Plateforme e-commerce centrale, Hybris est responsable de la création des commandes une fois le paiement validé. Après la soumission d'une commande par l'utilisateur, Hybris communique avec Adyen pour le traitement du paiement. Une fois la confirmation reçue, Hybris crée officiellement la commande et notifie Fluent pour la gestion de l'expédition et du suivi.
    \item [$\bullet$]\textbf{Adyen :} En tant que fournisseur de services de paiement, Adyen gère la transaction financière. Il traite les détails du paiement transmis par Hybris, réalisant des actions telles que l'autorisation, la capture des fonds, ou la gestion des abonnements. Adyen joue un rôle clé dans la sécurisation et la validation des paiements avant la finalisation de la commande.
    \item [$\bullet$]\textbf{Fluent :} Système de gestion logistique, Fluent est responsable du cycle de vie de la commande après sa création dans Hybris. Il assure le suivi du processus de traitement, y compris la préparation de l'expédition, et met à jour les statuts de la commande (par exemple, "CREATED", "PENDING PAYMENT", "CANCELLED"). Fluent communique également avec Hybris pour informer les utilisateurs de l'état de leur commande.
\end{itemize}
Pour mieux comprendre le fonctionnement de ces composants, examinons le flux du processus de paiement étape par étape :
\begin{enumerate}
    \item \textbf{Initiation :} Le client valide son panier et soumet sa commande.
    \item \textbf{Transmission :} Hybris communique les informations de paiement à Adyen.
    \item \textbf{Traitement :} Adyen exécute la transaction et renvoie une confirmation à Hybris.
    \item \textbf{Création :} Hybris enregistre la commande et notifie Fluent pour la gestion logistique.
    \item \textbf{Suivi :} Fluent assure la gestion des statuts de commande et informe Hybris pour la mise à jour du client.
\end{enumerate}
\section{Etude fonctionnelle et non fonctionnelle}
Dans le cadre de l'intégration de Payconiq à la plateforme e-commerce du client, il est essentiel de définir clairement les besoins fonctionnels et non fonctionnels du projet. 
Cette étude permettra d'identifier les exigences spécifiques liées à cette intégration, assurant ainsi une mise en œuvre réussie et une expérience utilisateur optimale. 

\subsection{Exigences fonctionnelles}

\subsubsection{Identification des fonctionnalités}
\begin{itemize}
    \item [$\bullet$]\textbf{Sélection de Payconiq comme méthode de paiement :} L'option Payconiq doit être disponible lors de la finalisation de la commande, présentée de manière intuitive et clairement distinguée parmi les autres choix de paiement.
    \item [$\bullet$]\textbf{Génération et affichage du QR Code pour le paiement :} Lors de la confirmation de la commande avec Payconiq, un QR code unique est généré automatiquement et affiché à l'utilisateur pour permettre un paiement sécurisé via l'application Payconiq.
    \item [$\bullet$]\textbf{Suivi des transactions dans l'historique des commandes :} Les transactions effectuées avec Payconiq sont accessibles dans l'historique des commandes de l'utilisateur, avec une identification claire comprenant la date, le montant et l'état du paiement.
    \item [$\bullet$]\textbf{Envoi d'un email de confirmation de commande :} Après la finalisation de la commande, un email de confirmation est envoyé à l'utilisateur, incluant les détails de la commande, la méthode de paiement utilisée (Payconiq), le montant total et les informations de suivi de la commande.
    \item [$\bullet$]\textbf{Message d'erreur en cas d'échec du paiement :} En cas d'échec du paiement avec Payconiq, un message d'erreur explicite informe l'utilisateur du problème, avec des options de paiement alternatives proposées pour finaliser la transaction sans interruption.
\end{itemize}

\subsubsection{Diagramme de cas d’utilisation}
\begin{center}
    \centering
    \includegraphics[width=19cm]{Figures/usecase.jpg}
    \captionof{figure}{Diagramme de cas d'utilisation}
    \label{fig:usecase}
\end{center}

\subsubsection[*]{Analyse des cas d'utilisation}


\subsubsection{Diagramme de séquence de système}

\subsection{Exigences non-fonctionnelles}

Les exigences non-fonctionnelles sont essentielles pour améliorer la qualité des services de la plateforme, notamment en termes de sécurité, de maintenabilité et de disponibilité. Elles garantissent non seulement une expérience utilisateur optimale, mais aussi la pérennité et l'efficacité du système face aux évolutions technologiques et aux besoins des utilisateurs. Parmi ces exigences, on peut citer :

\begin{itemize}
    \item [$\bullet$]\textbf{Sécurité :} La protection des informations sensibles des utilisateurs est cruciale dans tout système de paiement en ligne. L'intégration de Payconiq requiert une attention particulière à ces aspects pour éviter tout accès non autorisé et toute manipulation malveillante des données.
    \item [$\bullet$]\textbf{Maintenabilité :} La maintenabilité du système est essentielle pour garantir sa longévité et sa capacité à évoluer. Un code clair, bien structuré, et conforme aux meilleures pratiques de développement facilite l'ajout de nouvelles fonctionnalités et permet une correction rapide des bugs. Une bonne maintenabilité permet aux équipes de développement de diagnostiquer et de résoudre efficacement les problèmes, de déployer des mises à jour sans perturber le fonctionnement du système, et d'adapter rapidement le système aux évolutions technologiques et aux besoins des utilisateurs.
    \item [$\bullet$]\textbf{Disponibilité :} Le système doit être conçu pour faciliter le diagnostic, la résolution des problèmes, le déploiement des mises à jour, et l’adaptation aux évolutions technologiques et aux besoins des utilisateurs.
\end{itemize}


\section*{Conclusion}
La phase d'analyse de l'existant et de spécification des besoins est cruciale pour le succès de notre projet. Nous avons abordé cette phase en examinant l'architecture actuelle, en identifiant les acteurs clés et les cas d'utilisation spécifiques au marché belge. Ensuite nous avons entamé l'analyse des besoins fonctionnels et non fonctionnels. Dans les chapitres suivants, nous aborderons la conception détaillée du projet d'intégration.

\chapter{Conception de la solution}
\label{Conception de la solution}

Ce chapitre se concentre sur l'architecture physique et applicative adoptée. Il inclut également une exploration approfondie de la conception, illustrée par des diagrammes techniques de classes et de séquences détaillées, en se basant sur les spécifications établies dans le deuxième chapitre.
\newpage


\section{Architecture Globale}

L'architecture e-commerce, telle que présentée ici, est volontairement simplifiée pour se concentrer sur les composants essentiels sur lesquels nous avons travaillé. Elle s’appuie sur plusieurs composants clés pour gérer les interactions entre le client, le contenu du site, les commandes et les paiements.
\begin{center}
    \centering
    \includegraphics[width=19cm]{Figures/architectureGlobale.png}
    \captionof{figure}{Architecture globale de l'application E-commerce}
    \label{fig:processus}
\end{center}
\begin{itemize}
    \item [$\bullet$]Le site web permet aux clients de naviguer et de consulter le contenu (produits, offres, etc.). Le contenu est géré par le module CMS de SAP Hybris, qui centralise les informations et les affiches sur le storefront.
    \item [$\bullet$]Le moteur de recherche Fredhopper se charge de fournir les résultats de recherche et de navigation aux utilisateurs, en se basant sur les catalogues et les offres définis via PEARL, un système de gestion de l'information produit (PIM).
    \item [$\bullet$]Une fois la commande placée par le client via le site web, elle est prise en charge par SAP Hybris. Elle est ensuite transmise à Fluent Commerce, un système de gestion des commandes (OMS), qui suit et gère les différents statuts de la commande, depuis sa validation jusqu’à son expédition.
    \item [$\bullet$]CapAdresse est utilisé pour valider l’adresse du client avant la confirmation de la commande, afin de garantir la précision de la livraison.
    \item [$\bullet$]Mailjet est utilisé pour envoyer des notifications aux clients concernant les transactions, les newsletters, et d'autres communications, comme la confirmation de commande, d’expédition, ou d’annulation en cas de problème de paiement.
    \item [$\bullet$]Gigya s’occupe de la gestion des comptes clients, notamment l’enregistrement et la connexion des utilisateurs sur le site.
\end{itemize}
Lorsqu'un client effectue un paiement sur le site, une requête est envoyée à Adyen pour obtenir une autorisation. Adyen traite la demande et renvoie une réponse indiquant si le paiement est "autorisé" ou "non autorisé". Si le paiement est refusé, le client est redirigé vers une page d'erreur. En revanche, si le paiement est approuvé, le client est dirigé vers la page de confirmation de commande, et simultanément, la commande est exportée vers Fluent Commerce, le système de gestion des commandes.Fluent Commerce est chargé de gérer le cycle de vie de la commande en attendant la notification d'Adyen confirmant que le paiement a bien été autorisé, car il peut être annulé en cas de fraude ou de problème de sécurité. Si la notification indique que le paiement a été annulé, Fluent communique avec Mailjet pour envoyer un e-mail d'annulation au client. Cependant, si le paiement est confirmé, un e-mail de confirmation de commande est envoyé via Mailjet. Par la suite, une fois la commande prête à être expédiée, Fluent envoie une requête de capture à Adyen. Dès qu'Adyen confirme que la capture a été effectuée avec succès, la commande est expédiée, et Mailjet envoie un e-mail de confirmation de livraison au client. En plus de ces tâches, Fluent Commerce prend également en charge plusieurs aspects logistiques de la commande, tels que la vérification de la disponibilité des produits, la gestion des entrepôts pour l’expédition, ainsi que le suivi du statut de la commande (en attente, en préparation, expédiée, etc.).

\section{Architecture Backend}
Notre solution est basée sur la plate-forme SAP Hybris et suit donc son architecture
logicielle présentée dans la figure suivante :

\begin{center}
    \centering
    \includegraphics[width=19cm]{Figures/architecturebacend.png}
    \captionof{figure}{Architecture de backend de l'application E-commerce}
    \label{fig:processus}
\end{center}

\begin{itemize}
    \item [$\bullet$]\textbf{Serveur Apache Tomcat:} La plateforme Hybris utilise Apache Tomcat comme serveur HTTP intégré. Ce serveur d'application est responsable de l'hébergement de l'application Hybris et de la gestion des requêtes HTTP/HTTPS entrantes, assurant ainsi le bon fonctionnement de l'application web.
    \item [$\bullet$]\textbf{Extension Backoffice de Hybris:} L'extension Backoffice est une composante essentielle de Hybris qui permet aux utilisateurs métiers d'accéder aux fonctionnalités d'administration de contenu. Cela inclut la gestion du catalogue, des catégories, des produits, ainsi que des entités et types du système. Cette extension fournit une interface graphique conviviale pour les utilisateurs finaux, facilitant la gestion et l'organisation des données. En outre, elle offre aux développeurs la possibilité de créer ou de personnaliser des composants Hybris en fonction des besoins spécifiques de l'entreprise.
    \item [$\bullet$]\textbf{Service Layer de Hybris: }La couche de service (Service Layer) représente la couche métier de l'application Hybris, où est implémentée la logique de gestion d'entreprise. Elle est constituée d'un ensemble de services qui encapsulent les règles métiers et les processus de l'entreprise. La couche de service communique à la fois avec l'extension Backoffice et la couche de persistance via des modèles, qui sont des représentations des entités de la logique métier. Ces modèles servent d'intermédiaires entre les différentes couches, facilitant ainsi la manipulation des données de manière cohérente et sécurisée.
    \item [$\bullet$]\textbf{Couche de Persistance de Hybris:} La couche de persistance est le composant qui assure l'interaction entre le Service Layer et la base de données. Elle est responsable de la gestion de toutes les opérations de lecture et d'écriture dans la base de données, garantissant que les données sont stockées de manière efficace et peuvent être récupérées de manière fiable. 
\end{itemize}

\section{Environnent de livraison et test }


Cette section présente processus de déploiement du projet, en détaillant les différents serveurs utilisés. La structure se compose de deux parties principales : interne et externe.

\begin{center}
    \centering
    \includegraphics[width=19cm]{Figures/Test.png}
    \captionof{figure}{Architecture de livraison et test simplifiée}
    \label{fig:processus}
\end{center}
\begin{center}
    \centering
    \includegraphics[width=19cm]{Figures/UAT.png}
    \captionof{figure}{Environnements de test et production}
    \label{fig:processus}
\end{center}



\subsection{Architecture interne}
L'architecture interne est implémentée au sein de SQLI et comprend un serveur d'intégration (INT) et un serveur SonarQube pour l'analyse de la qualité du code. Les développeurs effectuent des commits sur le dépôt Azure DevOps, puis un serveur d'intégration continue, Jenkins, récupère les dernières versions depuis ce dépôt pour effectuer une compilation automatique avec ANT. Jenkins lance ensuite une analyse SonarQube pour garantir la qualité du code. Après l'analyse SonarQube, le serveur vérifie que le nombre d'erreurs détectées ne dépasse pas les limites prédéfinies en fonction de la gravité des erreurs et des quotas définis sur le serveur SonarQube. Si ces contrôles sont satisfaits, Jenkins déploie le code sur le serveur d'intégration (INT). Ce processus d'intégration continue se répète tout au long du sprint de trois semaines.


\subsection{Architecture externe}

L'architecture externe est implémentée dans l'environnement du client et se compose de trois serveurs : UAT (User Acceptance Testing), OAT (Operational Acceptance Testing) et PROD (Production). Toutes les deux semaines, une nouvelle version est livrée dans le système de gestion de versions du client, TFS (Team Foundation Server).

\begin{itemize} \item [$\bullet$]\textbf{Serveur UAT:} Utilisé pour tester les livrables et s'assurer qu'ils répondent aux attentes de l'utilisateur final. \item [$\bullet$]\textbf{Serveur OAT:} Permet de déterminer si les livrables sont opérationnels et prêts à être intégrés dans l'environnement de production. \item [$\bullet$]\textbf{Serveur PROD:} Après les processus de test, une mise en production (MEP) est effectuée tous les trois mois, rendant la version finale disponible pour les utilisateurs finaux. \end{itemize}

\section{Architecture applicative}

Dans le cadre de l'utilisation de la solution Hybris, il est conseillé de suivre l'architecture applicative qu'elle propose. Cette architecture est basée sur un modèle n-tiers largement utilisé dans les applications web, permettant une répartition claire des rôles et une meilleure structuration du code. Quatre principaux patrons de conception sont au cœur de cette architecture :

\begin{center}
    \centering
    \includegraphics[width=19cm]{Figures/Dto.png}
    \captionof{figure}{Architecture Applicative de l'application E-commerce}
    \label{fig:processus}
\end{center} 

\begin{itemize}
    \item[$\bullet$] \textbf{Modèle MVC (Modèle-Vue-Contrôleur)} : Ce modèle permet de séparer distinctement la présentation, la logique métier et l'accès aux données. Cela garantit une organisation modulaire, rendant l'application plus facile à maintenir et à étendre. Par exemple, dans notre application, la vue pourrait être gérée par des pages JSP ou des composants front-end, tandis que les contrôleurs orchestrent les opérations de la logique métier encapsulée dans les services.

    \item[$\bullet$] \textbf{Patron de Façade} : Ce patron vise à simplifier l'accès à un système complexe en fournissant une interface unique et uniforme. Il permet d'interagir facilement avec des sous-systèmes tout en masquant leur complexité interne. Dans notre projet, une façade serait utile pour centraliser les interactions entre les services et les DAO, en facilitant ainsi l'appel des contrôleurs.

    \item[$\bullet$] \textbf{Patron DAO (Data Access Object)} : Ce patron permet d'accéder aux données sans être lié à un SGBD spécifique, en fournissant une abstraction qui rend l'application plus flexible. Il encapsule la logique d'accès aux données, permettant d'intégrer facilement différents SGBD sans modifier le code de l'application. Par exemple, le DAO pour "Produit" ou "Client" peut contenir toute la logique nécessaire pour interagir avec les données associées, indépendamment de la base utilisée.

    \item[$\bullet$] \textbf{Patron DTO (Data Transfer Object)} : Ce modèle optimise les échanges de données entre différentes couches de l'application en regroupant les informations dans des objets spécifiques. Il permet de réduire la surcharge de transfert de données, améliorant ainsi l'efficacité des communications entre les couches. Dans notre application, les DTO servent de pont entre la couche de service et la présentation, réduisant les dépendances et simplifiant les tests unitaires.
\end{itemize}

Adopter cette architecture permet non seulement une meilleure organisation du code, mais aussi d'assurer une évolutivité et une maintenance aisée à long terme.

\section{Diagramme de classe}
\begin{center}
    \centering
    \includegraphics[width=19cm]{Figures/class.png}
    \captionof{figure}{Diagramme de classe}
\end{center}

La figure ci-dessus représente le diagramme de classe de notre système. Un Client peut passer plusieurs Commandes (représentées par la classe AbstractOrder), et chaque commande est associée à une Transaction de paiement (PaymentTransaction). Cette transaction contient des détails tels que le fournisseur de paiement, le montant prévu, et les identifiants du marchand. Elle est liée à une ou plusieurs Entrées de transaction (PaymentTransactionEntry), qui décrivent des aspects spécifiques comme le type de transaction, le montant, l'heure, et l'état de la transaction. Les informations de paiement sont représentées par la classe PaymentInfo, qui inclut le mode de paiement (défini par l'énumération PaymentModeType), et peut être associée à une Adresse. Une spécialisation de PaymentInfo, nommée PayconiqPaymentInfo, est utilisée pour les transactions via Payconiq. Le système gère également les devises via la classe Currency, qui est associée aux transactions de paiement. En somme, ce diagramme modélise la partie structurelle d'un volet de la gestion de paiement dans notre site e-commerce où les clients effectuent des paiements par divers moyens.

\section{Diagramme de séquence}

\begin{center}
    \centering
    \includegraphics[width=19cm]{Figures/sequence.png}
    \captionof{figure}{Diagramme de séquence }
    \label{fig:processus}
\end{center} 


Le processus de paiement via Payconiq commence lorsque le client sélectionne ce moyen de paiement sur la plateforme. La transaction est alors initialisée et validée pour s'assurer que toutes les conditions sont remplies. Ensuite, une session Payconiq est démarrée via Adyen, qui génère une URL de redirection pour amener le client à la page de paiement Payconiq. Une fois redirigé, une commande est créée avec le statut "en attente". La page de paiement affiche un QR code que le client doit scanner avec son application Payconiq. Après avoir scanné le QR code, la confirmation du paiement est envoyée de Payconiq à Adyen, qui informe ensuite la plateforme e-commerce du statut du paiement.

Si le paiement est autorisé, la commande est complétée avec le statut "créée", le client est redirigé vers une page de confirmation, et un e-mail de confirmation de commande est envoyé. En revanche, si le paiement échoue, la commande est marquée avec le statut "échec", le client est redirigé vers une page d'échec de paiement, et un e-mail d'échec de paiement est envoyé.



\subsection*{Conclusion}

Ce chapitre a été dédié à l’étude conceptuelle du projet. Après une présentation des architectures adoptées et des divers diagrammes techniques de classes et de séquences, une compréhension approfondie du projet a été acquise. La prochaine étape consistera à aborder l’implémentation et la validation de la solution, sujet du chapitre suivant.
\pagebreak

\chapter{Implémentation et Validation}
\label{chap:Implémentation et Validation}


Ce chapitre décrit l'implémentation du travail réalisé. Il commence par une présentation des technologies utilisées, suivie de captures d'écran illustrant les différentes fonctionnalités développées. Ensuite, les tests effectués sont exposés.
\newpage
\section{Technologies utilisées}
\section{Captures d'écran}
Cette partie fournit une vue détaillée des fonctionnalités développées à travers une série de captures d’écran.
Avant de finaliser l'intégration de Payconiq via Adyen, il est crucial de vérifier que cette méthode de paiement est correctement activée dans le système. Comme le montre la figure \ref{fig:activ}, Payconiq est marqué comme actif dans le Cockpit d'administration de SAP Commerce. 
Cela confirme qu'il est prêt à être utilisé pour le traitement des paiements, garantissant que les transactions effectuées avec Payconiq seront traitées correctement.
\begin{center}
    \centering
    \includegraphics[width=19cm]{Figures/Screens/VERIFIER QUE payment activer.png}
    \captionof{figure}{Activation de Payconiq}
    \label{fig:activ}
\end{center}
Une fois activée, Payconiq a été ajoutée à la boutique en ligne pour le marché belge. La figure \ref{fig:disp} montre que Payconiq apparaît désormais parmi les méthodes de paiement disponibles dans l'onglet eCommerce, aux côtés de giftcard, applepay-eu, et paypal-eu. Cette configuration permet aux utilisateurs de choisir Payconiq lors de la validation de leur commande.
\begin{center}
    \centering
    \includegraphics[width=19cm]{Figures/Screens/activation du payconiq pour belge.png}
    \captionof{figure}{Disponibilité de Payconiq dans la boutique en ligne}
    \label{fig:disp}
\end{center}
Lorsque le client sélectionne les articles désirés, il passe à la phase de paiement. La figure \ref{fig:selection} illustre l'ajout d'un produit au panier, une étape préalable au paiement.
\begin{center}
    \centering
    \includegraphics[width=19cm]{Figures/Screens/ajouter un produit au panier.png}
    \captionof{figure}{Ajout d'un produit au panier}
    \label{fig:selection}
\end{center}
C'est ici que commence la première étape du processus de paiement, où l'utilisateur est invité à saisir ses informations personnelles, y compris l'adresse e-mail, comme le montre la figure \ref{fig:saisie}.
\begin{center}
    \centering
    \includegraphics[width=19cm]{Figures/Screens/ajout de l'email.png}
    \captionof{figure}{Fournir les informations personnelles}
    \label{fig:saisie}
\end{center}
Après avoir rempli ses informations personnelles, l'utilisateur passe à la deuxième étape du processus de paiement, dédiée à la sélection du mode de livraison. Comme illustré dans la figure \ref{fig:mode}, cette étape permet à l'utilisateur de choisir entre la livraison à domicile ou le retrait en magasin (Click \& Collect). 
En fonction de l'option sélectionnée, l'interface offre les champs nécessaires : pour la livraison à domicile, l'utilisateur doit fournir une adresse complète, tandis que pour le retrait en magasin, il choisit le point de retrait souhaité. Cette personnalisation assure une expérience de commande adaptée aux préférences de livraison de chaque utilisateur.
\begin{center}
    \centering
    \includegraphics[width=19cm]{Figures/Screens/Infos livraison.png}
    \captionof{figure}{Mode de livraison}
    \label{fig:mode}
\end{center}
Si un code promotionnel est disponible, l'utilisateur peut l'entrer dans le champ prévu à cet effet (\textit{Figure \ref{fig:promo}}) pour bénéficier d'une réduction. 
\begin{center}
    \centering
    \includegraphics[width=19cm]{Figures/Screens/code prommo.png}
    \captionof{figure}{Saisir un code promo}
    \label{fig:promo}
\end{center}
En l'absence de code, l'utilisateur poursuit directement vers l'étape suivante : la facturation. À ce stade, il est invité à saisir ses informations de facturation. L'interface (\textit{Figure \ref{fig:facturation}}) propose de reprendre automatiquement l'adresse de livraison pour simplifier le processus, mais permet également de renseigner une adresse différente si nécessaire. Cette flexibilité assure que les informations de facturation peuvent être adaptées selon les besoins de l'utilisateur, tout en garantissant que les détails de facturation restent distincts de ceux de la livraison, si tel est le souhait.
\begin{center}
    \centering
    \includegraphics[width=19cm]{Figures/Screens/passe au facturation.png}
    \captionof{figure}{Facturation}
    \label{fig:facturation}
\end{center}
Après que le client a complété toutes ses informations de facturation, il accède à l'interface de sélection des modes de paiement(\textit{Figure \ref{fig:mode_paiement}}). À cette étape, il peut choisir parmi plusieurs options disponibles, y compris Payconiq, qui est mise en avant comme méthode de paiement. 
L'interface est conçue pour faciliter le choix de la méthode préférée par l'utilisateur avant de procéder à l'étape suivante du processus de paiement.
\begin{center}
    \centering
    \includegraphics[width=19cm]{Figures/Screens/payment.png}
    \captionof{figure}{Choix de mode de paiement}
    \label{fig:mode_paiement}
\end{center}
Après avoir sélectionné le mode de paiement, l'utilisateur est dirigé vers l'API Adyen. Sur cette interface, un QR code est généré pour finaliser la transaction. Comme l'illustre la figure \ref{fig:qr}, ce QR code doit être scanné dans les 15 minutes pour éviter l'annulation automatique de la commande. L'interface fournit également un compte à rebours indiquant le temps restant pour effectuer le paiement, garantissant ainsi une expérience utilisateur fluide et sécurisée.
\begin{center}
    \centering
    \includegraphics[width=19cm]{Figures/Screens/redirection.png}
    \captionof{figure}{Finalisation du paiement via l'API Adyen}
    \label{fig:qr}
\end{center}
Une fois redirigé vers la page de paiement, l'utilisateur traverse plusieurs étapes :
Tout d'abord, il est accueilli par un écran indiquant qu'il doit patienter pendant que la transaction est traitée. 
\begin{center}
    \centering
    \includegraphics[width=10cm]{Figures/Screens/patience.jpeg}
    \captionof{figure}{Écran de traitement en cours}
    \label{fig:patience}
\end{center}
Ensuite, l'écran suivant affiche le montant total à payer et demande a l'utilisateur d'autoriser la transaction, comme le montre la figure \ref{fig:autorisation}
\begin{center}
    \centering
    \includegraphics[width=10cm]{Figures/Screens/montant.jpeg}
    \captionof{figure}{Montant à payer}
    \label{fig:autorisation}
\end{center}
Enfin, lorsque la transaction est réussie, l'utilisateur est redirigé vers un écran de confirmation de paiement, illustré dans la figure \ref{fig:reussie}
\begin{center}
    \centering
    \includegraphics[width=10cm]{Figures/Screens/reussi.jpeg}
    \captionof{figure}{Paiement reussi}
    \label{fig:reussie}
\end{center}
Après la finalisation réussie du paiement, l'utilisateur est dirigé vers une page de confirmation de commande (\textit{Figure \ref{fig:confirmation}}). Cette page présente un identifiant unique de suivi, facilitant le suivi de l'état de la commande. 
\begin{center}
    \centering
    \includegraphics[width=19cm]{Figures/Screens/confirmation du commande.png}
    \captionof{figure}{Confirmation de la commande}
    \label{fig:confirmation}
\end{center}
En plus de cet identifiant, toutes les informations nécessaires sont fournies (\textit{Figure \ref{fig:resume}}), telles que l'email pour le suivi de la livraison. L'interface propose également des options pour gérer la commande, notamment la possibilité de créer un compte utilisateur pour un accès facilité aux informations de commande et aux fonctionnalités associées.
\begin{center}
    \centering
    \includegraphics[width=19cm]{Figures/Screens/resumer commande.png}
    \captionof{figure}{Informations de la commande}
    \label{fig:resume}
\end{center}
En cas d'annulation ou de non-approbation du paiement, la page de confirmation de commande affichera un message d'erreur détaillé. Comme le montre la figure \ref{fig:erreur}, un avertissement est clairement indiqué : "Remarque : les informations de paiement sont erronées. Veuillez vérifier ces informations afin de confirmer votre commande." Ce message signifie que le processus de validation du paiement n'a pas été complété correctement, nécessitant une révision des informations saisies pour procéder à la confirmation de la commande.
\begin{center}
    \centering
    \includegraphics[width=19cm]{Figures/Screens/annulation de commande.png}
    \captionof{figure}{Message d'erreur en cas d'annulation du paiement}
    \label{fig:erreur}
\end{center}
Si la commande est effectuée avec succès, la figure \ref{fig:admin} illustre l'interface backoffice qui confirme la création de la commande et la validation réussie du paiement. Dans cet écran, l'administrateur peut consulter les détails de la commande, s'assurer que le paiement a été correctement traité et apporter toute modification nécessaire. Cette étape est essentielle pour le suivi du processus de commande et permet de gérer efficacement les commandes après l'autorisation du paiement.
\begin{center}
    \centering
    \includegraphics[width=19cm]{Figures/Screens/Backoffice commande.png}
    \captionof{figure}{Confirmation de la commande dans le backoffice}
    \label{fig:admin}
\end{center}
Une fois la commande validée, un e-mail est automatiquement envoyé à l'utilisateur, confirmant les détails de l'achat et assurant une traçabilité complète. Ce message inclut des informations essentielles telles que le numéro de commande, les articles achetés, le montant total, ainsi que les coordonnées de livraison. Cette étape permet à l'utilisateur de suivre sa commande de manière transparente et de vérifier les détails importants relatifs à son achat.
\begin{center}
    \centering
    \includegraphics[width=19cm]{Figures/Screens/confirmation par mail.png}
    \captionof{figure}{Message de confirmation par email}
    \label{fig:email}
\end{center}




















\section{Tests et Validation}


\section*{Conclusion}
Ce chapitre a été consacré à la mise en place de la solution développée. Après une brève introduction des technologies employées, les captures d'écran ont mis en lumière les fonctionnalités réalisées. Les détails des tests entrepris ont ensuite permis de valider l'efficacité de la nouvelle méthode de paiement. Ces étapes ont assuré que la plateforme répond aux exigences fonctionnelles et maintient une performance optimale.




% \input{Chapters/Chapter1}

% \input{Chapters/Chapter2}

% \input{Chapters/Chapter3}

% \input{Chapters/Chapter4}

% \input{Chapters/Chapter5}


% \chapter*{Conclusion générale}
\addcontentsline{toc}{chapter}{Conclusion}


Ce rapport a détaillé les résultats de mon projet de fin d’études, centré sur l'amélioration d'une plateforme e-commerce par l’intégration de la méthode de paiement Payconiq, spécialement adaptée au marché belge. L’objectif principal était de moderniser la plateforme pour optimiser l’expérience utilisateur et répondre aux exigences locales.

Au cours du projet, j’ai exploré divers aspects du développement de systèmes, tels que la planification technique, la conception, le déploiement et les tests. Cette expérience m’a permis d’appliquer les connaissances acquises à l’INPT tout en découvrant de nouvelles technologies dans un environnement professionnel. Mon rôle au sein de l’équipe de développement a été particulièrement enrichissant, m’offrant une perspective concrète sur la gestion de projets réels.

Bien que mon implication dans ce projet se termine ici, plusieurs pistes d’amélioration pour la plateforme e-commerce restent ouvertes. Il serait pertinent de poursuivre l’optimisation des processus de paiement, d’intégrer de nouvelles méthodes de paiement et de renforcer la sécurité. Ces développements permettront à la plateforme de rester compétitive et de s’adapter aux évolutions du marché et des technologies.

Les compétences acquises lors de ce projet seront précieuses pour mes futurs projets, notamment dans le cadre de mes prochaines missions avec SQLi. Ce projet de fin d’études a non seulement permis d'améliorer la plateforme e-commerce, mais a également constitué une étape significative dans ma formation, me préparant pour de nouveaux défis professionnels.








\appendix


% \input{Endmatter/Glossary}

\chapter*{Conclusion générale}
\addcontentsline{toc}{chapter}{Conclusion}


Ce rapport a détaillé les résultats de mon projet de fin d’études, centré sur l'amélioration d'une plateforme e-commerce par l’intégration de la méthode de paiement Payconiq, spécialement adaptée au marché belge. L’objectif principal était de moderniser la plateforme pour optimiser l’expérience utilisateur et répondre aux exigences locales.

Au cours du projet, j’ai exploré divers aspects du développement de systèmes, tels que la planification technique, la conception, le déploiement et les tests. Cette expérience m’a permis d’appliquer les connaissances acquises à l’INPT tout en découvrant de nouvelles technologies dans un environnement professionnel. Mon rôle au sein de l’équipe de développement a été particulièrement enrichissant, m’offrant une perspective concrète sur la gestion de projets réels.

Bien que mon implication dans ce projet se termine ici, plusieurs pistes d’amélioration pour la plateforme e-commerce restent ouvertes. Il serait pertinent de poursuivre l’optimisation des processus de paiement, d’intégrer de nouvelles méthodes de paiement et de renforcer la sécurité. Ces développements permettront à la plateforme de rester compétitive et de s’adapter aux évolutions du marché et des technologies.

Les compétences acquises lors de ce projet seront précieuses pour mes futurs projets, notamment dans le cadre de mes prochaines missions avec SQLi. Ce projet de fin d’études a non seulement permis d'améliorer la plateforme e-commerce, mais a également constitué une étape significative dans ma formation, me préparant pour de nouveaux défis professionnels.








% \input{Endmatter/ComplementaryCodes}


\renewcommand{\bibname}{Références}

\addcontentsline{toc}{chapter}{Références}
\begin{thebibliography}{99}
    \bibitem{SQLI}
    \emph{SQLI},
    \href{}{\textbf{https://fr.4d.com/.}}
    
    \bibitem{valeurSQLI}
    \emph{les valeurs du groupe SQLI },
    \href{https://www.sqli.com/ma-fr/carriere/les-valeurs-sqli}{\textbf{https://www.sqli.com/ma-fr/carriere/les-valeurs-sqli}}
    
    \bibitem{4Dhistory}
    \emph{About 4D},
    \href{https://www.sqli.com/ma-fr}{\textbf{https://www.sqli.com/ma-fr}}
    
    \bibitem{4DleaderShip}
    \emph{4D GROUP LEADERSHIP},
    \href{https://uk.4d.com/leadership/}{\textbf{https://uk.4d.com/leadership/}}
    
  
     %the link is a documentation of the basic bibliography method (that    I'm using here) + bibTex which is more advanced, read it well and decide which one works best for you.
     \bibitem{UML}
     \emph{Qu'est-ce que le langage UML },
     \href{https://www.lucidchart.com/pages/fr/langage-uml}{\textbf{https://www.lucidchart.com/pages/fr/langage-uml}}
     
     \bibitem{Typescript}
     \emph{TypeScript pour les Programmeurs JavaScript},
     \href{https://www.typescriptlang.org/fr/docs/handbook/typescript-in-5-minutes.html}{\textbf{document}}
     
     \bibitem{VS}
     \emph{Visual Studio },
     \href{https://code.visualstudio.com/}{\textbf{https://code.visualstudio.com/}}
     
     \bibitem{Postman}
     \emph{Postman },
     \href{https ://www.postman.com}{\textbf{https ://www.postman.com}}
     
     \bibitem{GitLab}
     \emph{GitLab},
     \href{https://about.gitlab.com/}{\textbf{https://about.gitlab.com/}}
     
     \bibitem{StarUml}
     \emph{StarUML},
     \href{https://inf1410.teluq.ca/teluqDownload.php?file=2014/01/INF1410-PresentationStarUML.pdf}{\textbf{Document}}
     

     \bibitem{axios}
     \emph{Axios},
     \href{https://axios-http.com/fr/docs/intro}{\textbf{https://axios-http.com/fr/docs/intro}}
     
    \end{thebibliography}

% \begin{thebibliography}{99}
% \addcontentsline{toc}{chapter}{Bibliography}


% \bibitem{ref1}
% Author name, Book name.

% \bibitem{ref2}
% \emph{Title 1},
% \href{https://www.overleaf.com/learn/latex/Bibliography_management_with_bibtex}{\textbf{Title 2}}

%  %the link is a documentation of the basic bibliography method (that    I'm using here) + bibTex which is more advanced, read it well and decide which one works best for you.



% \end{thebibliography}

\end{document}