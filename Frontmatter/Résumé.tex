

\chapter*{Résumé}
\addcontentsline{toc}{chapter}{Résumé}

Dans un contexte où le commerce électronique est en constante évolution, les entreprises doivent régulièrement mettre à jour et optimiser leurs plateformes pour rester compétitives. L'accélération des innovations technologiques et les attentes croissantes des consommateurs exigent une adaptation continue pour offrir des expériences utilisateur de haute qualité et répondre aux nouvelles exigences du marché. \\
\vspace{10pt}


Ce rapport présente un aperçu concis de mes contributions lors de mon stage de fin d'études effectué au sein de SQLI, dans le but d'obtenir le titre d'Ingénieur d'État en Télécommunications et Technologies de l'Information, spécialisation en Ingénierie Logicielle Avancée pour les Services Numériques à l'Institut National des Postes et Télécommunications. Ce projet avait pour objectif d'améliorer une plateforme e-commerce existante pour un client, en intégrant la méthode de paiement Payconiq spécifiquement pour le marché belge, tout en corrigeant divers bugs afin d'optimiser la performance du système.\\
\vspace{10pt}

En prenant en compte la complexité de SAP Hybris, la réalisation de ce projet a nécessité l'utilisation de technologies telles que JEE et Spring pour le développement, ainsi que JUnit pour le testing. L'intégration de Payconiq a été un défi majeur, nécessitant une attention particulière à la compatibilité et à la performance pour garantir une solution adaptée aux besoins spécifiques du marché belge.
\vspace{10pt}

\noindent\rule[2pt]{\textwidth}{0.5pt}

{\textbf{Mots clés :}}
E-commerce, Payconiq, SAP Hybris, JUnit, JEE, Spring, Déboggage
\\
\noindent\rule[2pt]{\textwidth}{0.5pt}