\chapter*{Introduction générale}
\addcontentsline{toc}{chapter}{Introduction générale}


Dans un contexte numérique en constante évolution, où les attentes des consommateurs sont de plus en plus exigeantes, il devient impératif pour les entreprises de perfectionner leurs plateformes e-commerce afin de garantir une expérience utilisateur fluide et moderne. Ce projet de fin d’études s’inscrit dans cette dynamique, en cherchant à renforcer les performances d’une plateforme existante tout en intégrant de nouvelles fonctionnalités adaptées au marché belge.

\vspace{10pt}

L’objectif central de ce projet est d’améliorer la plateforme en y intégrant la méthode de paiement Payconiq, une solution largement utilisée en Belgique. Cependant, l'intégration de cette nouvelle méthode de paiement ne constitue qu'une partie de l'amélioration. Mon travail a également consisté à détecter et corriger des anomalies au sein de la plateforme, contribuant ainsi à une meilleure stabilité, performance, et satisfaction client.

Cette démarche a nécessité une approche combinant analyse technique, identification des points faibles existants et développement de nouvelles solutions, avec pour finalité d’assurer une expérience utilisateur optimisée et conforme aux besoins du marché local. 

Ce rapport se compose de quatre chapitres distincts :

Le premier chapitre présente le contexte général du projet, en détaillant l'organisme d'accueil, la problématique identifiée, les objectifs fixés, ainsi que la méthodologie de travail adoptée.

Le deuxième chapitre est consacré à l'analyse de l'existant et à l'étude des besoins fonctionnels et non fonctionnels, en mettant en évidence les spécificités actuelles de la plateforme et les exigences liées à l'intégration de Payconiq.

Le troisième chapitre se concentre sur la conception du projet, illustrée par des diagrammes UML qui décrivent les structures et les interactions au sein de la plateforme.


Enfin, le quatrième chapitre traite la réalisation du projet, en détaillant non seulement le processus d’intégration de la nouvelle méthode de paiement, mais aussi les corrections apportées aux bugs et l’évaluation de leur impact sur la plateforme.
\vspace{10pt}


