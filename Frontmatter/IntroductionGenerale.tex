\chapter*{Introduction générale}
\addcontentsline{toc}{chapter}{Introdcution}


Dans un environnement numérique en perpétuelle évolution, où les attentes des consommateurs et les technologies avancent rapidement, il est crucial pour les entreprises d'améliorer continuellement leurs plateformes e-commerce afin de rester compétitives. Ce projet de fin d'études s'inscrit dans ce contexte dynamique, en visant à améliorer une plateforme e-commerce pour un client spécifique et à intégrer la méthode de paiement Payconiq, particulièrement pertinente pour le marché belge.

\vspace{10pt}

L’objectif principal de ce projet est de moderniser la plateforme en ajoutant Payconiq comme nouvelle méthode de paiement, afin d'optimiser l'expérience utilisateur et de répondre aux exigences locales. Ce rapport se compose de quatre chapitres distincts : 

Le premier chapitre présente le contexte général du projet, en détaillant l'organisme d'accueil, l'équipe impliquée, la problématique identifiée, les objectifs visés, ainsi que la méthodologie de travail adoptée. 

Le deuxième chapitre est dédié à l'analyse de l'existant et à l'étude des besoins fonctionnels et non fonctionnels, en mettant en lumière les spécificités actuelles de la plateforme et les exigences liées à l'intégration de Payconiq.

Le troisième chapitre se concentre sur la conception du projet, illustrée par les diagrammes UML qui décrivent les structures et les interactions au sein de la plateforme.

Enfin, le quatrième chapitre traite de la réalisation du projet, en détaillant le processus d'implémentation de la nouvelle méthode de paiement et en évaluant son impact sur la plateforme.

\vspace{10pt}

Ce rapport fournit une vue d'ensemble complète de l'amélioration de la plateforme e-commerce et de l'intégration de Payconiq, tout en offrant des perspectives pour les développements futurs.

