\chapter*{\RL{ملخص}}
\addcontentsline{toc}{chapter}{Arabic Abstract}

\begin{RLtext}

 \noindent\hspace{10pt} 
 
 في ظل التطور المستمر الذي يشهده مجال التجارة الإلكترونية، تجد الشركات نفسها مضطرة لتحديث وتحسين منصاتها بشكل منتظم للحفاظ على قدرتها التنافسية. إذ أن تسارع الابتكارات التكنولوجية وتزايد توقعات المستهلكين يفرضان تكيفًا مستمرًا من أجل تقديم تجارب مستخدم عالية الجودة وتلبية المتطلبات الجديدة للسوق.

 \vspace{10pt}

لذا فهذا التقرير يقدم نظرة موجزة عن مساهماتي التي أجريتها في شركة \LR{SQLI} في هذا الصدد خلال فترة التدريب النهائي لنيل شهادة مهندس دولة في الاتصالات وتكنولوجيا المعلومات، بتخصص في هندسة البرمجيات المتقدمة للخدمات الرقمية من المعهد الوطني للبريد والاتصالات. كان الهدف من هذا المشروع تحسين منصة تجارة إلكترونية قائمة لأحد العملاء، من خلال دمج نظام الدفع \LR{Payconiq} خصيصًا للسوق البلجيكي، مع تصحيح مجموعة من الأخطاء البرمجية لتحسين أداء النظام.
 \vspace{10pt}

 لتحقيق النجاح في هذا المشروع، بدأنا بتحليل متعمق للمشروع بهدف تحديد الاحتياجات الوظيفية والفنية التي يجب أن يلبيها الحل. وفي الخطوة الثانية، أجرينا دراسة مفاهيمية مترجمة إلى رسوم بيانية. وأخيراً، بدأنا في تطوير الحل وتنفيذه.

 \vspace{10pt}

 تم تنفيذ المشروع باستخدام React Framework على الجانب الأمامي و4D للواجهة الخلفية.

 \vspace{10pt}


\end{RLtext}

\noindent\rule[2pt]{\textwidth}{0.5pt}

\begin{RLtext}
    {\textbf{الكلمات المفتاحية}}
    \end{RLtext}
    \hfill
    React, TypeScript, 4D, E-learning, Tailwind
    



\noindent\rule[2pt]{\textwidth}{0.5pt}