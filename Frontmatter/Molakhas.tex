\chapter*{\begin{RLtext}{ملخص}\end{RLtext}}
\addcontentsline{toc}{chapter}{Résumé arabe}

\begin{RLtext}
 \noindent\hspace{10pt} 
 في ظل التطور المستمر الذي يشهده مجال التجارة الإلكترونية، تجد الشركات نفسها مضطرة لتحديث وتحسين منصاتها بشكل منتظم للحفاظ على قدرتها التنافسية. إذ أن تسارع الابتكارات التكنولوجية وتزايد توقعات المستهلكين يفرضان تكيفًا مستمرًا من أجل تقديم تجارب استخدام عالية الجودة وتلبية المتطلبات الجديدة للسوق.

 \vspace{10pt}

 لذا فهذا التقرير يقدم نظرة موجزة عن مساهماتي التي قدمتها في شركة \LR{"SQLI Maroc"} خلال فترة التدريب النهائي لنيل شهادة مهندس دولة في الاتصالات وتكنولوجيا المعلومات، بتخصص في هندسة البرمجيات المتقدمة للخدمات الرقمية من المعهد الوطني للبريد والمواصلات. كان الهدف من هذا المشروع تحسين منصة تجارة إلكترونية قائمة لأحد العملاء، من خلال دمج نظام الدفع \LR{"Payconiq"} خصيصًا للسوق البلجيكي، مع تصحيح مجموعة من الأخطاء البرمجية لتحسين أداء النظام.
 \vspace{10pt}

 ونظراً لتعقيد منصة \LR{"SAP Hybris"}، تطلب تنفيذ هذا المشروع استخدام تقنيات مثل \LR{"JEE"} و\LR{"Spring"} للتطوير، بالإضافة إلى \LR{"JUnit"} للاختبار. شكلت عملية دمج \LR{"Payconiq"} تحديًا رئيسيا، حيث استلزمت اهتماما خاصا بالتوافق والأداء لضمان تقديم حل يلبي الاحتياجات المحددة للسوق البلجيكي.

 \vspace{10pt}
\end{RLtext}

\noindent\rule[2pt]{\textwidth}{0.5pt}

\noindent
\begin{RLtext}
    \textbf{الكلمات المفتاحية\LR{:}} التجارة الإلكترونية،\LR{Payconiq}، \LR{SAP Hybris}، \LR{JUnit}، \LR{JEE}، \LR{Spring}، \LR{debugging}.
\end{RLtext}

\noindent\rule[2pt]{\textwidth}{0.5pt}
