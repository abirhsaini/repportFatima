\chapter*{Conclusion générale}
\addcontentsline{toc}{chapter}{Conclusion}
Ce rapport a détaillé les résultats de mon projet de fin d’études, axé sur l'amélioration d'une plateforme e-commerce en intégrant la méthode de paiement Payconiq, spécifiquement pour le marché belge. L’objectif principal de ce projet était de moderniser la plateforme en ajoutant cette méthode de paiement pour optimiser l’expérience utilisateur et répondre aux exigences locales.

Tout au long de ce projet, j’ai exploré divers aspects du développement du système, y compris la planification technique, la conception, le déploiement et les tests rigoureux. Ce projet m’a permis de mettre en pratique les connaissances et compétences acquises au cours de mes études à l'INPT, tout en découvrant de nouvelles technologies et en intégrant un environnement professionnel. Mon expérience au sein de l’équipe de développement a été particulièrement enrichissante, offrant une perspective précieuse sur le travail en milieu professionnel et la gestion de projets réels.

Pour l’avenir, plusieurs perspectives guideront le développement ultérieur de la plateforme e-commerce et l’intégration de nouvelles fonctionnalités. Les efforts futurs pourraient inclure l’optimisation continue des processus de paiement, le renforcement des mesures de sécurité et l’exploration de nouvelles méthodes de paiement adaptées à différents marchés. Ces améliorations visent à garantir que la plateforme reste compétitive, sécurisée et adaptée aux besoins des utilisateurs, tout en répondant aux évolutions du marché et des technologies.

En conclusion, ce projet a non seulement permis d'améliorer la plateforme e-commerce en intégrant Payconiq, mais a également offert une opportunité précieuse pour affiner mes compétences techniques et professionnelles. 