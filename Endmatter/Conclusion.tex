\chapter*{Conclusion générale}
\addcontentsline{toc}{chapter}{Conclusion}


Ce rapport a détaillé les résultats de mon projet de fin d’études, centré sur l'amélioration d'une plateforme e-commerce par l’intégration de la méthode de paiement Payconiq, spécialement adaptée au marché belge. L’objectif principal était de moderniser la plateforme pour optimiser l’expérience utilisateur et répondre aux exigences locales.

Au cours du projet, j’ai exploré divers aspects du développement de systèmes, tels que la planification technique, la conception, le déploiement et les tests. Cette expérience m’a permis d’appliquer les connaissances acquises à l’INPT tout en découvrant de nouvelles technologies dans un environnement professionnel. Mon rôle au sein de l’équipe de développement a été particulièrement enrichissant, m’offrant une perspective concrète sur la gestion de projets réels.

Bien que mon implication dans ce projet se termine ici, plusieurs pistes d’amélioration pour la plateforme e-commerce restent ouvertes. Il serait pertinent de poursuivre l’optimisation des processus de paiement, d’intégrer de nouvelles méthodes de paiement et de renforcer la sécurité. Ces développements permettront à la plateforme de rester compétitive et de s’adapter aux évolutions du marché et des technologies.

Les compétences acquises lors de ce projet seront précieuses pour mes futurs projets, notamment dans le cadre de mes prochaines missions avec SQLi. Ce projet de fin d’études a non seulement permis d'améliorer la plateforme e-commerce, mais a également constitué une étape significative dans ma formation, me préparant pour de nouveaux défis professionnels.






