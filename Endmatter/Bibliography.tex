
\renewcommand{\bibname}{Références}

\addcontentsline{toc}{chapter}{Références}
\begin{thebibliography}{99}
    \bibitem{SQLI}
    \emph{SQLI},
    \href{}{\textbf{https://fr.4d.com/.}}
    
    \bibitem{valeurSQLI}
    \emph{les valeurs du groupe SQLI },
    \href{https://www.sqli.com/ma-fr/carriere/les-valeurs-sqli}{\textbf{https://www.sqli.com/ma-fr/carriere/les-valeurs-sqli}}
    
    \bibitem{4Dhistory}
    \emph{About 4D},
    \href{https://www.sqli.com/ma-fr}{\textbf{https://www.sqli.com/ma-fr}}
    
    \bibitem{4DleaderShip}
    \emph{4D GROUP LEADERSHIP},
    \href{https://uk.4d.com/leadership/}{\textbf{https://uk.4d.com/leadership/}}
    
  
     %the link is a documentation of the basic bibliography method (that    I'm using here) + bibTex which is more advanced, read it well and decide which one works best for you.
     \bibitem{UML}
     \emph{Qu'est-ce que le langage UML },
     \href{https://www.lucidchart.com/pages/fr/langage-uml}{\textbf{https://www.lucidchart.com/pages/fr/langage-uml}}
     
     \bibitem{Typescript}
     \emph{TypeScript pour les Programmeurs JavaScript},
     \href{https://www.typescriptlang.org/fr/docs/handbook/typescript-in-5-minutes.html}{\textbf{document}}
     
     \bibitem{VS}
     \emph{Visual Studio },
     \href{https://code.visualstudio.com/}{\textbf{https://code.visualstudio.com/}}
     
     \bibitem{Postman}
     \emph{Postman },
     \href{https ://www.postman.com}{\textbf{https ://www.postman.com}}
     
     \bibitem{GitLab}
     \emph{GitLab},
     \href{https://about.gitlab.com/}{\textbf{https://about.gitlab.com/}}
     
     \bibitem{StarUml}
     \emph{StarUML},
     \href{https://inf1410.teluq.ca/teluqDownload.php?file=2014/01/INF1410-PresentationStarUML.pdf}{\textbf{Document}}
     

     \bibitem{axios}
     \emph{Axios},
     \href{https://axios-http.com/fr/docs/intro}{\textbf{https://axios-http.com/fr/docs/intro}}
     
    \end{thebibliography}