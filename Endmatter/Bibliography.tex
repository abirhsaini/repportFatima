
\renewcommand{\bibname}{Références}

\addcontentsline{toc}{chapter}{Références}
\begin{thebibliography}{99}
    \bibitem{SQLI}
    \emph{SQLI},\\
    \href{https://www.sqli.com/ma-fr}{\textbf{https://www.sqli.com/ma-fr}}
    
    \bibitem{valeurSQLI}
    \emph{les valeurs du groupe SQLI },\\
    \href{https://www.sqli.com/ma-fr/carriere/les-valeurs-sqli}{\textbf{https://www.sqli.com/ma-fr/carriere/les-valeurs-sqli}}
    
    \bibitem{Belge}
    \emph{Utilisateur de Payconiq}, \\
    \href{https://assets-eu-01.kc-usercontent.com/331a0185-e26c-01d0-41e9-c42bc79a1d2c/6bf42db6-6363-4041-b9d3-54c6548f8e14/Press%20Release-22Feb2024.pdf}{\textbf{https://assets-eu-01.kc-usercontent.com/331a0185-e26c-01d0-41e9-c42bc79a1d2c/6bf42db6-6363-4041-b9d3-54c6548f8e14/Press\%20Release-22Feb2024.pdf}}

    \bibitem{Payconic}
    \emph{ Justification de l’intégration de Bancontact by Payconiq},\\
    \href{https://www.payconiq.be/fr/professionnel/pourquoi-choisir-payconiq}{\textbf{https://www.payconiq.be/fr/professionnel/pourquoi-choisir-payconiq}}

    \bibitem{SAP}
    \emph{SAP},\\
    \href{https://help.sap.com/docs/}{\textbf{https://help.sap.com/docs/}}

    \bibitem{JEE} 
    \emph{JEE}, \\
    \href{https://www.neosoft.fr/nos-publications/blog-tech/les-bases-de-la-securite-en-developpement-java-ee/}{\textbf{https://www.neosoft.fr/nos-publications/blog-tech/les-bases-de-la-securite-en-developpement-java-ee/}}
   

    \bibitem{spring} 
    \emph{Spring Framework Documentation}, \\
    \href{https://docs.spring.io/spring-framework/reference/index.html}{\textbf{https://docs.spring.io/spring-framework/reference/index.html}}

    \bibitem{sonarqube} 
    \emph{SonarQube Documentation}, \\
    \href{https://docs.sonarsource.com/sonarqube/latest/}{\textbf{https://docs.sonarsource.com/sonarqube/latest/}}

    \bibitem{ant} 
    \emph{Apache Ant Manual}, \\
    \href{https://ant.apache.org/manual/index.html}{\textbf{https://ant.apache.org/manual/index.html}}

    \bibitem{junit} 
    \emph{JUnit 4 Documentation}, \\
    \href{https://junit.org/junit4/javadoc/latest/}{\textbf{https://junit.org/junit4/javadoc/latest/}}

    %  %the link is a documentation of the basic bibliography method (that    I'm using here) + bibTex which is more advanced, read it well and decide which one works best for you.
    %  \bibitem{UML}
    %  \emph{Qu'est-ce que le langage UML },
    %  \href{https://www.lucidchart.com/pages/fr/langage-uml}{\textbf{https://www.lucidchart.com/pages/fr/langage-uml}}
     
    %  \bibitem{Typescript}
    %  \emph{TypeScript pour les Programmeurs JavaScript},
    %  \href{https://www.typescriptlang.org/fr/docs/handbook/typescript-in-5-minutes.html}{\textbf{document}}
     
    %  \bibitem{VS}
    %  \emph{Visual Studio },
    %  \href{https://code.visualstudio.com/}{\textbf{https://code.visualstudio.com/}}
     
    %  \bibitem{Postman}
    %  \emph{Postman },
    %  \href{https ://www.postman.com}{\textbf{https ://www.postman.com}}
     
    %  \bibitem{GitLab}
    %  \emph{GitLab},
    %  \href{https://about.gitlab.com/}{\textbf{https://about.gitlab.com/}}
     
    %  \bibitem{StarUml}
    %  \emph{StarUML},
    %  \href{https://inf1410.teluq.ca/teluqDownload.php?file=2014/01/INF1410-PresentationStarUML.pdf}{\textbf{Document}}
     

    %  \bibitem{axios}
    %  \emph{Axios},
    %  \href{https://axios-http.com/fr/docs/intro}{\textbf{https://axios-http.com/fr/docs/intro}}
     
    \end{thebibliography}