\chapter{Implémentation et Validation}
\label{chap:Implémentation et Validation}


Ce chapitre décrit l'implémentation du travail réalisé. Il commence par une présentation des technologies utilisées, suivie de captures d'écran illustrant les différentes fonctionnalités développées. Ensuite, les tests effectués sont exposés.
\newpage
\section{Technologies utilisées}
\subsection*{SAP Commerce (Hybris)}
\begin{center}
    \centering
    \includegraphics[scale=0.5]{Figures/SAP.jpg}
    \captionof{figure}{Logo de SAP }
    \label{fig:processus}
\end{center} 

SAP Commerce (Hybris) est une plateforme de commerce électronique robuste et évolutive, conçue pour répondre aux besoins des entreprises dans un environnement omnicanal. Elle permet de gérer efficacement les catalogues de produits, les commandes, les paiements, ainsi que les interactions clients via divers canaux de vente tels que le web, les appareils mobiles, les points de vente physiques et autres. De plus, elle intègre des outils de suivi des performances et de marketing pour une gestion complète du commerce en ligne, ce qui justifie son choix dans notre projet.

\subsection{JEE (JSP)}
\begin{center}
    \centering
    \includegraphics[scale=0.5]{Figures/jee.png}
    \captionof{figure}{Logo de JEE}
    \label{fig:processus}
\end{center} 
Nous avons utilisé JEE (JSP) pour ajouter la partie frontend liée à notre méthode de paiement. Cette technologie nous a permis de créer des pages web dynamiques en combinant du code Java et HTML, ce qui facilite l'affichage interactif des informations de paiement et la gestion des transactions utilisateur. Grâce à JSP, nous avons pu développer une interface utilisateur réactive tout en gardant la flexibilité et la puissance du backend Java pour assurer la bonne gestion des flux de paiement.Nous avons utilisé JEE (JSP) pour ajouter la partie frontend liée à notre méthode de paiement. Cette technologie nous a permis de créer des pages web dynamiques en combinant du code Java et HTML, ce qui facilite l'affichage interactif des informations de paiement et la gestion des transactions utilisateur. Grâce à JSP, nous avons pu développer une interface utilisateur réactive tout en gardant la flexibilité et la puissance du backend Java pour assurer la bonne gestion des flux de paiement.

\subsection*{Spring}
\begin{center}
    \centering
    \includegraphics[scale=0.4]{Figures/spring.png}
    \captionof{figure}{Logo de Spring}
    \label{fig:processus}
\end{center} 

Dans le cadre de notre projet, Spring a été utilise pour développer la logique métier associée à la nouvelle méthode de paiement. Cela a impliqué la création de services back-end pour la validation des paiements, l'interaction avec les systèmes de paiement externes, et le traitement des retours d'information, garantissant ainsi une séparation claire des responsabilités et une maintenabilité accrue du code. Le choix de Spring s’explique par le fait que SAP Commerce (Hybris), sur lequel repose notre projet, est construit sur ce framework. Cette compatibilité assure une intégration fluide avec Hybris et permet de tirer parti des fonctionnalités avancées de Spring en matière de gestion des transactions et de sécurité, essentielles pour la fiabilité et la performance de la méthode de paiement.
\subsection*{SonarQube}
 \begin{center}
    \centering
    \includegraphics[scale=0.15]{Figures/sonarqube.png}
    \captionof{figure}{Logo de SonarQube}
    \label{fig:processus}
\end{center} 

Sonar est un outil d’analyse statique de code qui aide à évaluer la qualité du code source. Il détecte les bugs, les vulnérabilités de sécurité, et fournit des recommandations pour améliorer la lisibilité et la maintenabilité du code. L'utilisation de Sonar est souvent associée à des pratiques de développement telles que l'intégration continue pour garantir que le code livré soit de haute qualité.

Dans le cadre de ce projet, Sonar a été utilisé pour analyser à la fois le code existant et les nouvelles fonctionnalités ajoutées. Cela nous a permis de corriger les bugs et les failles de sécurité identifiés par l'outil, et également d'utiliser les indicateurs de qualité fournis pour améliorer la structure et la lisibilité du code. Cette approche a garanti que le code respectait les standards de qualité du client.

\subsection*{Ant}
\begin{center}
    \centering
    \includegraphics[scale=0.05]{Figures/Ant.png}
    \captionof{figure}{Logo d'Ant}

\end{center}
Ant, un outil de build open-source largement utilisé dans les projets Java, permet d’automatiser diverses tâches telles que la compilation, la création de fichiers exécutables, le déploiement et l'exécution des tests. Dans notre projet, Ant a été utilisé pour automatiser le processus de compilation, effectuer les tests unitaires, et faciliter le débogage. En raison de sa configuration native avec Hybris, qui repose sur des scripts Ant prédéfinis, notre choix d'Ant s'est avéré particulièrement adapté. Cela nous a permis d'organiser efficacement ces différentes étapes, assurant ainsi un flux de développement plus fluide et bien structuré.

\subsection*{JUnit}
\begin{center}
    \centering
    \includegraphics[scale=0.5]{Figures/junit.png}
    \captionof{figure}{Logo de JUnit}
    \label{fig:processus}
\end{center}
JUnit est un framework de tests unitaires en Java qui permet de tester les composants individuels d'une application pour s'assurer qu'ils fonctionnent comme prévu. Il est essentiel dans le processus de développement, car il permet d'identifier et de corriger rapidement les bugs tout en garantissant que le code fonctionne correctement après chaque modification.

Dans le cadre de l'intégration de la nouvelle méthode de paiement, j'ai écrit des tests unitaires en utilisant JUnit pour vérifier la fiabilité des fonctionnalités back-end. J'ai testé des scénarios comme la validation des transactions, la gestion des erreurs, et les interactions avec les systèmes externes. Grâce à l'exigence du client d'une couverture de 100% des tests, nous avons veillé à ce que chaque composant du système soit minutieusement testé, garantissant ainsi que toutes les fonctionnalités répondaient aux spécifications et étaient exemptes de bugs.
\section{Illustration de l'Intégration de la Méthode de Paiement}
Cette partie fournit une vue détaillée des fonctionnalités développées à travers une série de captures d’écran.
Avant de finaliser l'intégration de Payconiq via Adyen, il est crucial de vérifier que cette méthode de paiement est correctement activée dans le système. Comme le montre la figure \ref{fig:activ}, Payconiq est marqué comme actif dans le Cockpit d'administration de SAP Commerce. 
Cela confirme qu'il est prêt à être utilisé pour le traitement des paiements, garantissant que les transactions effectuées avec Payconiq seront traitées correctement.
\begin{center}
    \centering
    \includegraphics[width=19cm]{Figures/Screens/VERIFIER QUE payment activer.png}
    \captionof{figure}{Activation de Payconiq}
    \label{fig:activ}
\end{center}
Une fois activée, Payconiq a été ajoutée à la boutique en ligne pour le marché belge. La figure \ref{fig:disp} montre que Payconiq apparaît désormais parmi les méthodes de paiement disponibles dans l'onglet eCommerce, aux côtés de giftcard, applepay-eu, et paypal-eu. Cette configuration permet aux utilisateurs de choisir Payconiq lors de la validation de leur commande.
\begin{center}
    \centering
    \includegraphics[width=19cm]{Figures/Screens/activation du payconiq pour belge.png}
    \captionof{figure}{Disponibilité de Payconiq dans la boutique en ligne}
    \label{fig:disp}
\end{center}
Lorsque le client sélectionne les articles désirés, il passe à la phase de paiement. La figure \ref{fig:selection} illustre l'ajout d'un produit au panier, une étape préalable au paiement.
\begin{center}
    \centering
    \includegraphics[width=19cm]{Figures/Screens/ajouter un produit au panier.png}
    \captionof{figure}{Ajout d'un produit au panier}
    \label{fig:selection}
\end{center}
C'est ici que commence la première étape du processus de paiement, où l'utilisateur est invité à saisir ses informations personnelles, y compris l'adresse e-mail, comme le montre la figure \ref{fig:saisie}.
\begin{center}
    \centering
    \includegraphics[width=19cm]{Figures/Screens/ajout de l'email.png}
    \captionof{figure}{Fournir les informations personnelles}
    \label{fig:saisie}
\end{center}
Après avoir rempli ses informations personnelles, l'utilisateur passe à la deuxième étape du processus de paiement, dédiée à la sélection du mode de livraison. Comme illustré dans la figure \ref{fig:mode}, cette étape permet à l'utilisateur de choisir entre la livraison à domicile ou le retrait en magasin (Click \& Collect). 
En fonction de l'option sélectionnée, l'interface offre les champs nécessaires : pour la livraison à domicile, l'utilisateur doit fournir une adresse complète, tandis que pour le retrait en magasin, il choisit le point de retrait souhaité. Cette personnalisation assure une expérience de commande adaptée aux préférences de livraison de chaque utilisateur.
\begin{center}
    \centering
    \includegraphics[width=19cm]{Figures/Screens/Infos livraison.png}
    \captionof{figure}{Mode de livraison}
    \label{fig:mode}
\end{center}
Si un code promotionnel est disponible, l'utilisateur peut l'entrer dans le champ prévu à cet effet (\textit{Figure \ref{fig:promo}}) pour bénéficier d'une réduction. 
\begin{center}
    \centering
    \includegraphics[width=19cm]{Figures/Screens/code prommo.png}
    \captionof{figure}{Saisir un code promo}
    \label{fig:promo}
\end{center}
En l'absence de code, l'utilisateur poursuit directement vers l'étape suivante : la facturation. À ce stade, il est invité à saisir ses informations de facturation. L'interface (\textit{Figure \ref{fig:facturation}}) propose de reprendre automatiquement l'adresse de livraison pour simplifier le processus, mais permet également de renseigner une adresse différente si nécessaire. Cette flexibilité assure que les informations de facturation peuvent être adaptées selon les besoins de l'utilisateur, tout en garantissant que les détails de facturation restent distincts de ceux de la livraison, si tel est le souhait.
\begin{center}
    \centering
    \includegraphics[width=19cm]{Figures/Screens/passe au facturation.png}
    \captionof{figure}{Facturation}
    \label{fig:facturation}
\end{center}
Après que le client a complété toutes ses informations de facturation, il accède à l'interface de sélection des modes de paiement(\textit{Figure \ref{fig:mode_paiement}}). À cette étape, il peut choisir parmi plusieurs options disponibles, y compris Payconiq, qui est mise en avant comme méthode de paiement. 
L'interface est conçue pour faciliter le choix de la méthode préférée par l'utilisateur avant de procéder à l'étape suivante du processus de paiement.
\begin{center}
    \centering
    \includegraphics[width=19cm]{Figures/Screens/payment.png}
    \captionof{figure}{Choix de mode de paiement}
    \label{fig:mode_paiement}
\end{center}
Après avoir sélectionné le mode de paiement, l'utilisateur est dirigé vers l'API Adyen. Sur cette interface, un QR code est généré pour finaliser la transaction. Comme l'illustre la figure \ref{fig:qr}, ce QR code doit être scanné dans les 15 minutes pour éviter l'annulation automatique de la commande. L'interface fournit également un compte à rebours indiquant le temps restant pour effectuer le paiement, garantissant ainsi une expérience utilisateur fluide et sécurisée.
\begin{center}
    \centering
    \includegraphics[width=19cm]{Figures/Screens/redirection.png}
    \captionof{figure}{Finalisation du paiement via l'API Adyen}
    \label{fig:qr}
\end{center}
Une fois redirigé vers la page de paiement, l'utilisateur traverse plusieurs étapes :
Tout d'abord, il est accueilli par un écran indiquant qu'il doit patienter pendant que la transaction est traitée. 
\begin{center}
    \centering
    \includegraphics[width=10cm]{Figures/Screens/patience.jpeg}
    \captionof{figure}{Écran de traitement en cours}
    \label{fig:patience}
\end{center}
Ensuite, l'écran suivant affiche le montant total à payer et demande a l'utilisateur d'autoriser la transaction, comme le montre la figure \ref{fig:autorisation}
\begin{center}
    \centering
    \includegraphics[width=10cm]{Figures/Screens/montant.jpeg}
    \captionof{figure}{Montant à payer}
    \label{fig:autorisation}
\end{center}
Enfin, lorsque la transaction est réussie, l'utilisateur est redirigé vers un écran de confirmation de paiement, illustré dans la figure \ref{fig:reussie}
\begin{center}
    \centering
    \includegraphics[width=10cm]{Figures/Screens/reussi.jpeg}
    \captionof{figure}{Paiement reussi}
    \label{fig:reussie}
\end{center}
Après la finalisation réussie du paiement, l'utilisateur est dirigé vers une page de confirmation de commande (\textit{Figure \ref{fig:confirmation}}). Cette page présente un identifiant unique de suivi, facilitant le suivi de l'état de la commande. 
\begin{center}
    \centering
    \includegraphics[width=19cm]{Figures/Screens/confirmation du commande.png}
    \captionof{figure}{Confirmation de la commande}
    \label{fig:confirmation}
\end{center}
En plus de cet identifiant, toutes les informations nécessaires sont fournies (\textit{Figure \ref{fig:resume}}), telles que l'email pour le suivi de la livraison. L'interface propose également des options pour gérer la commande, notamment la possibilité de créer un compte utilisateur pour un accès facilité aux informations de commande et aux fonctionnalités associées.
\begin{center}
    \centering
    \includegraphics[width=19cm]{Figures/Screens/resumer commande.png}
    \captionof{figure}{Informations de la commande}
    \label{fig:resume}
\end{center}
En cas d'annulation ou de non-approbation du paiement, la page de confirmation de commande affichera un message d'erreur détaillé. Comme le montre la figure \ref{fig:erreur}, un avertissement est clairement indiqué : "Remarque : les informations de paiement sont erronées. Veuillez vérifier ces informations afin de confirmer votre commande." Ce message signifie que le processus de validation du paiement n'a pas été complété correctement, nécessitant une révision des informations saisies pour procéder à la confirmation de la commande.
\begin{center}
    \centering
    \includegraphics[width=19cm]{Figures/Screens/annulation de commande.png}
    \captionof{figure}{Message d'erreur en cas d'annulation du paiement}
    \label{fig:erreur}
\end{center}
Si la commande est effectuée avec succès, la figure \ref{fig:admin} illustre l'interface backoffice qui confirme la création de la commande et la validation réussie du paiement. Dans cet écran, l'administrateur peut consulter les détails de la commande, s'assurer que le paiement a été correctement traité et apporter toute modification nécessaire. Cette étape est essentielle pour le suivi du processus de commande et permet de gérer efficacement les commandes après l'autorisation du paiement.
\begin{center}
    \centering
    \includegraphics[width=19cm]{Figures/Screens/Backoffice commande.png}
    \captionof{figure}{Confirmation de la commande dans le backoffice}
    \label{fig:admin}
\end{center}
Une fois la commande validée, un e-mail est automatiquement envoyé à l'utilisateur, confirmant les détails de l'achat et assurant une traçabilité complète. Ce message inclut des informations essentielles telles que le numéro de commande, les articles achetés, le montant total, ainsi que les coordonnées de livraison. Cette étape permet à l'utilisateur de suivre sa commande de manière transparente et de vérifier les détails importants relatifs à son achat.
\begin{center}
    \centering
    \includegraphics[width=19cm]{Figures/Screens/confirmation par mail.png}
    \captionof{figure}{Message de confirmation par email}
    \label{fig:email}
\end{center}
\section{Analyse et Correction des Anomalies}

Au cours de cette période, une partie essentielle de mon travail a consisté à résoudre les problèmes détectés par SonarQube, un outil d'analyse statique qui permet d'améliorer la qualité du code. Comme illustré par les captures ci-dessous, SonarQube a mis en évidence plusieurs anomalies critiques, notamment des  duplications de littéraux  et des  complexités cognitives  élevées dans certaines méthodes du projet. Ces problèmes, jugés critiques, nécessitaient une attention immédiate pour garantir une bonne maintenabilité du projet.

En corrigeant ces anomalies, j'ai non seulement amélioré la maintenabilité du projet, mais aussi réduit les risques d'erreurs futures. Ce travail d'optimisation a permis de renforcer la  robustesse  du code et d'assurer que le projet respecte les bonnes pratiques de développement logiciel, en ligne avec les normes de qualité du secteur.
\begin{center}
    \centering
    \includegraphics[width=19cm]{Figures/Screens/bug.png}
    \captionof{figure}{Bugs détectés par SonarQube}
    \label{fig:corr}
\end{center}
La figure \ref{fig:ticket} illustre un exemple de ticket de bug  créé pour suivre les anomalies détectées. Ce processus de gestion des bugs est essentiel pour identifier et résoudre les problèmes techniques, ce qui garantit la  stabilité  et la  qualité  du produit final. Travailler sur ces tickets a également permis de faciliter la collaboration entre les équipes, améliorant ainsi l'expérience utilisateur globale.
\begin{center}
    \centering
    \includegraphics[width=19cm]{Figures/Screens/ticket.png}
    \captionof{figure}{Exemple ticket de debugging}
    \label{fig:ticket}
\end{center}
\section{Tests et Validation}
\subsection{Tests}
Les tests constituent une étape essentielle pour s'assurer du bon fonctionnement du système et de sa conformité aux attentes. Ils aident à repérer et corriger d'éventuelles anomalies avant la mise en production. Nous avons ainsi mené différents types de tests, allant des tests unitaires aux tests fonctionnels, réalisés sur plusieurs environnements :
\begin{itemize}
    \item[$\bullet$] \textbf{Tests unitaires :} Nous avons utilisé JUnit pour mener les tests unitaires, visant à valider chaque composant du système de manière isolée. Ces tests, réalisés en local, permettent de vérifier que les méthodes fonctionnent comme prévu, de gérer les exceptions, et d'assurer la robustesse du code avant toute intégration.
    \item[$\bullet$] \textbf{Tests fonctionnels :} Les tests fonctionnels ont été pris en charge par l'équipe QA sur plusieurs environnements, tels que INT2, INT1, UAT et OAT. Ces tests permettent de valider que les fonctionnalités développées respectent les exigences métiers et opèrent correctement dans des conditions réalistes. INT1 est utilisé comme environnement partagé par toutes les équipes, tandis que INT2 est dédié à l'équipe Cart \& Checkout Payment. Ces validations sont cruciales pour garantir la stabilité et la conformité du système avant sa mise en production.
\end{itemize}

\subsection{Validation des besoins fonctionnels}
Les besoins fonctionnels ont été validés avec succès sur l'environnement INT1. À ce stade, nous attendons la finalisation des tests dans tous les environnements pour pouvoir procéder au déploiement de la fonctionnalité en production. En parallèle, une validation finale est effectuée par le Delivery Manager afin de garantir que les objectifs fonctionnels et les exigences métiers sont entièrement satisfaits avant la mise en production.

Il est important de noter que la validation des besoins par le client est une étape cruciale qui n'a pas encore été réalisée. Nous planifions d'organiser une session de validation avec le client pour obtenir leur approbation formelle des besoins fonctionnels. Cette validation permettra de confirmer que les exigences sont bien comprises et acceptées avant le passage en production, garantissant ainsi que les attentes du client sont pleinement respectées.

\subsection{Validation des besoins non fonctionnels}
Pour garantir que le système répond aux critères essentiels de performance, il est crucial de valider les exigences non fonctionnelles définies préalablement.
\begin{itemize}
    \item[$\bullet$]\textbf{Sécurité :} La sécurité des paiements est garantie par l'utilisation de l'API d'Adyen, qui intègre des mécanismes de sécurité avancés. Adyen utilise des protocoles de cryptage de bout en bout pour protéger les données sensibles des transactions. En outre, l'API d'Adyen respecte les exigences de la directive européenne PSD2 en offrant une authentification forte du client (SCA). Cette authentification peut inclure des facteurs biométriques, des mots de passe ou des codes à usage unique, assurant ainsi une protection renforcée contre les fraudes et garantissant un niveau de sécurité optimal pour chaque transaction traitée.
    \item[$\bullet$] \textbf{Maintenabilité :} La maintenabilité du système est assurée grâce à une architecture modulaire et une documentation complète, facilitant la gestion des mises à jour et des correctifs. Les processus de déploiement automatisés minimisent les risques d'erreurs et permettent une intégration fluide des nouvelles fonctionnalités. Cette approche garantit une gestion efficace des évolutions et des corrections nécessaires au bon fonctionnement du système.
    \item[$\bullet$] \textbf{Disponibilité :} La disponibilité du système est optimisée par la configuration de déploiements répartis sur plusieurs clusters géographiques. Le projet One utilise cinq clusters situés en Amérique (AMER), en Europe, au Moyen-Orient et en Afrique (EMEA), en Asie-Pacifique, en Australie et au Canada (APAC), en Russie et en Chine. Cette répartition permet d'assurer une haute disponibilité et une résilience accrue du système, offrant une réponse rapide aux demandes des utilisateurs, indépendamment de leur emplacement. Grâce à cette architecture distribuée, le système bénéficie d'une latence réduite et d'un temps de réponse optimisé. En cas de panne d'un cluster, le trafic est automatiquement redirigé vers les autres clusters, assurant ainsi la continuité du service et minimisant les interruptions. Cette approche contribue à maintenir une expérience utilisateur fluide et efficace.
\end{itemize}

\section*{Conclusion}
Ce chapitre a été consacré à la mise en place de la solution développée. Après une brève introduction des technologies employées, les captures d'écran ont mis en lumière les fonctionnalités réalisées. Les détails des tests entrepris ont ensuite permis de valider l'efficacité de la nouvelle méthode de paiement. Ces étapes ont assuré que la plateforme répond aux exigences fonctionnelles et maintient une performance optimale.


